\documentclass[12pt, a4paper,oneside, UTF8]{ctexbook}
\usepackage[dvipsnames]{xcolor}
\usepackage{mathtools}   % 数学公式
\usepackage{amsthm}    % 定理环境
\usepackage{amssymb}   % 更多公式符号
\usepackage{graphicx}  % 插图
%\usepackage{mathrsfs}  % 数学字体
%\usepackage{newtxtext,newtxmath}
%\usepackage{arev}
\usepackage{kmath,kerkis}
\usepackage{newtxtext}
\usepackage{bbm}
\usepackage{enumitem}  % 列表
\usepackage{geometry}  % 页面调整
%\usepackage{unicode-math}
\usepackage[colorlinks,linkcolor=black]{hyperref}

\usepackage{wrapfig}


\usepackage{ulem}	   % 用于更多的下划线格式,
					   % \uline{}下划线,\uuline{}双下划线,\uwave{}下划波浪线,\sout{}中间删除线,\xout{}斜删除线
					   % \dashuline{}下划虚线,\dotuline{}文字底部加点


\graphicspath{ {flg/},{../flg/}, {config/}, {../config/} }  % 配置图形文件检索目录
\linespread{1.5} % 行高

% 页码设置
\geometry{top=25.4mm,bottom=25.4mm,left=20mm,right=20mm,headheight=2.17cm,headsep=4mm,footskip=12mm}

% 设置列表环境的上下间距
\setenumerate[1]{itemsep=5pt,partopsep=0pt,parsep=\parskip,topsep=5pt}
\setitemize[1]{itemsep=5pt,partopsep=0pt,parsep=\parskip,topsep=5pt}
\setdescription{itemsep=5pt,partopsep=0pt,parsep=\parskip,topsep=5pt}

% 定理环境
% ########## 定理环境 start ####################################
\theoremstyle{definition}
\newtheorem{defn}{\indent 定义}[section]

\newtheorem{lemma}{\indent 引理}[section]    % 引理 定理 推论 准则 共用一个编号计数
\newtheorem{thm}[lemma]{\indent 定理}
\newtheorem{corollary}[lemma]{\indent 推论}
\newtheorem{criterion}[lemma]{\indent 准则}

\newtheorem{proposition}{\indent 命题}[section]
\newtheorem{example}{\indent \color{SeaGreen}{例}}[section] % 绿色文字的 例 ,不需要就去除\color{SeaGreen}{}
\newtheorem*{rmk}{\indent \color{red}{注}}

% 两种方式定义中文的 证明 和 解 的环境:
% 缺点:\qedhere 命令将会失效【技术有限,暂时无法解决】
\renewenvironment{proof}{\par\textbf{证明.}\;}{\qed\par}
\newenvironment{solution}{\par{\textbf{解.}}\;}{\qed\par}

% 缺点:\bf 是过时命令,可以用 textb f等替代,但编译会有关于字体的警告,不过不影响使用【技术有限,暂时无法解决】
%\renewcommand{\proofname}{\indent\bf 证明}
%\newenvironment{solution}{\begin{proof}[\indent\bf 解]}{\end{proof}}
% ######### 定理环境 end  #####################################

% ↓↓↓↓↓↓↓↓↓↓↓↓↓↓↓↓↓ 以下是自定义的命令  ↓↓↓↓↓↓↓↓↓↓↓↓↓↓↓↓

% 用于调整表格的高度  使用 \hline\xrowht{25pt}
\newcommand{\xrowht}[2][0]{\addstackgap[.5\dimexpr#2\relax]{\vphantom{#1}}}

% 表格环境内长内容换行
\newcommand{\tabincell}[2]{\begin{tabular}{@{}#1@{}}#2\end{tabular}}

% 使用\linespread{1.5} 之后 cases 环境的行高也会改变,重新定义一个 ca 环境可以自动控制 cases 环境行高
\newenvironment{ca}[1][1]{\linespread{#1} \selectfont \begin{cases}}{\end{cases}}
% 和上面一样
\newenvironment{vx}[1][1]{\linespread{#1} \selectfont \begin{vmatrix}}{\end{vmatrix}}

\def\d{\textup{d}} % 直立体 d 用于微分符号 dx
\def\R{\mathbb{R}} % 实数域
\def\N{\mathbb{N}} % 自然数域
\def\C{\mathbb{C}} % 复数域
\def\Z{\mathbb{Z}} % 整数环
\def\Q{\mathbb{Q}} % 有理数域
\newcommand{\bs}[1]{\boldsymbol{#1}}    % 加粗,常用于向量
\newcommand{\ora}[1]{\overrightarrow{#1}} % 向量

% 数学 平行 符号
\newcommand{\pll}{\kern 0.56em/\kern -0.8em /\kern 0.56em}

% 用于空行\myspace{1} 表示空一行 填 2 表示空两行  
\newcommand{\myspace}[1]{\par\vspace{#1\baselineskip}}

%s.t. 用\st就能打出s.t.
\DeclareMathOperator{\st}{s.t.}

%罗马数字 \rmnum{}是小写罗马数字, \Rmnum{}是大写罗马数字
\makeatletter
\newcommand{\rmnum}[1]{\romannumeral #1}
\newcommand{\Rmnum}[1]{\expandafter@slowromancap\romannumeral #1@}
\makeatother