\ifx\allfiles\undefined
\input{../config/config}
\begin{document}
	% \input{../config/cover} 
	\else
	\fi
	%  ############################ 正文部分
\chapter{赋范空间}
\section{赋范线性空间}
\subsection{赋范线性空间}
	这一章我们将来讨论\textbf{赋范线性空间}的相关定义及性质. 首先回顾\textbf{范数}的定义\textbf{(同定义 \ref{def 1.1.1})}.
	
	\begin{defn}\label{def 2.1.1}
		Let $X$ be a vector space over field $\mathbb{F}$, a \underline{\textcolor{blue}{\textbf{norm}}} is a function:
		\begin{align}
			X &\longrightarrow \R_{\geq 0} \\
			f &\longmapsto \Vert f \Vert
		\end{align}
		satisfying the following properties:
		\begin{enumerate}
			\item[(\rmnum{1})]$\Vert f \Vert \geq 0$, $\forall f \in X$. \hspace*{3em} ($\Vert f \Vert = 0 \,\, \Leftrightarrow \,\, f = 0$)
			
			\item[(\rmnum{2})]$\Vert af \Vert = \left| a \right| \Vert f \Vert$, $\forall a \in \mathbb{F}, f \in X$.
			
			\item[(\rmnum{3})]$\Vert f + g \Vert \leq \Vert f \Vert + \Vert g \Vert$, $\forall f , g \in X$.
		\end{enumerate}
		
		\vspace*{0.5em}
		
		\begin{rmk}
			\begin{itemize}
				\item 事实上\textbf{赋范线性空间}的称呼有些许不妥, 应直接称\textbf{赋范空间}. 因为\textbf{范数}的概念本就需要在\textbf{线性空间}上定义. 
				\begin{center}
					(否则\textbf{范数定义第(\rmnum{3})条“三角不等式”}中“$f + g$” 的加法无从定义)
				\end{center}
				
				\vspace*{2em}
				
				\item \textbf{赋范空间}与\textbf{度量空间}的关系为:
				\begin{center}
					\textbf{赋范空间} $\,\, \Rightarrow \,\,$ \textbf{度量空间} 
					\hspace*{1em} , \hspace*{1em} 
					\textbf{度量空间} $\,\, \not\Rightarrow \,\,$ \textbf{赋范空间}
				\end{center}
				即\textbf{赋范空间}上均可定义度量, 但\textbf{度量空间}却不一定能扩充为\textbf{赋范空间}. \\ 其中对于任意\textbf{赋范空间}$(X , \Vert \cdot \Vert)$, 其上总是默认定义如下\textbf{度量$d$}:
				\[ d(x , y) = \Vert x - y \Vert , \,\, \forall x , y \in X \]
				不难证明其满足\textbf{度量}的三条公理 (\textbf{Def \ref{def 1.1.2}})
				
				\newpage
				
				\item 对于“\textbf{度量空间} $\,\, \not\Rightarrow \,\,$ \textbf{赋范空间}”, 先来明确\textbf{度量空间$(X , \rho)$}要扩充为\textbf{赋范空间}所需条件:
				
				\vspace*{1em}
				
				\begin{enumerate}
					\item 引入(数)域$\mathbb{F}$, 在$X$ 中定义加法$\&$ 数乘运算, 使其满足线性空间八大公理. 即度量$\rho$ 需要先定义在线性空间上, 使其称为\textbf{度量线性空间}. 
					
					\vspace*{0.5em}
					
					\item 度量$\rho$ 需要满足\textbf{平移不变性}, 即
					\[ \rho(x , 0) = \rho(x + y , y) , \,\, \forall x , y \in X \]
					
					\item 对应\textbf{范数}定义的\textbf{绝对齐性}, 度量$\rho$ 也需要满足
					\[ \rho(\alpha x , 0) = \left| \alpha \right| \rho(x , 0) , \,\, \forall x \in X , \alpha \in \mathbb{F} \]
				\end{enumerate}
				
				\vspace*{1em}
				
				而即便是对于\textbf{度量线性空间}, 大多数也并不满足后两者条件, 下面给出一个反例.
				
				\begin{example}\label{ex 2.1.1}
					在欧氏空间$\R$ 中定义度量$\rho$:
					\[ \rho(x , y) = \frac{\left| x - y \right|}{1 + \left| x - y \right|} , \,\, \forall x , y \in \R \]
					则$\rho$ 不满足绝对齐性, 从而无法扩充为\textbf{赋范空间}.
				\end{example}
				
				\vspace*{6em}
				
				\item 对于任意\textbf{赋范空间}$(X , \Vert \cdot \Vert)$, 其上定义的范数$\Vert \cdot \Vert$ 均为连续函数.
				
				\begin{itemize}
					\item 
					\begin{proof}
						下面分两步进行证明, 首先证明一条引理. 
						\begin{enumerate}
							\item $\forall x , y \in X$, $\Big| \, \Vert x \Vert - \Vert y \Vert \, \Big| \leq \Vert x - y \Vert$:\\
							根据范数的三角不等式, $\forall x , y \in X$,
							\[ \Vert x \Vert \leq \Vert x - y \Vert + \Vert y \Vert \]
							\[ \Vert y \Vert \leq \Vert y - x \Vert + \Vert x \Vert = \Vert x - y \Vert \]
							移项后可得:$\Big| \, \Vert x \Vert - \Vert y \Vert \, \Big| \leq \Vert x - y \Vert$.
							
							\vspace*{2em}
							
							\item $\Vert \cdot \Vert$ 连续:$\forall \{ x_n \}_{n = 1}^{\infty} \subset X$ with $x_n \to x \in X$. Since
							\[ \Big| \, \Vert x_n \Vert - \Vert x \Vert \, \Big| \leq \Vert x_n - x \Vert \]
							Thus $\Vert x_n \Vert \to \Vert x \Vert$, 即$\Vert \cdot \Vert$ 连续.
						\end{enumerate}
					\end{proof}
				\end{itemize}
			\end{itemize}
		\end{rmk}
	\end{defn}

\newpage

\subsection{Banach空间}
	下面给出\textbf{Banach空间}的定义.
	
	\begin{defn}\label{def 2.1.2}
		完备的$B^*$ 空间称为\underline{\textcolor{blue}{\textbf{$B$ 空间}}} / \underline{\textcolor{blue}{\textbf{Banach空间}}}. 
		
		\vspace*{2em}
		
		\begin{rmk}
			\begin{itemize}
				\item 我们称\textbf{赋范空间}为\underline{\textcolor{blue}{\textbf{$B^*$ 空间}}}. 
				
				\vspace*{1em}
				
				\item 若$X$ 为\textbf{赋范空间}, 我们记$X \in B^*$;若$X$ 为\textbf{Banach空间}, 我们记$X \in B$.
			\end{itemize}
		\end{rmk}
	\end{defn}






	%  ############################
	\ifx\allfiles\undefined
\end{document}
\fi