\ifx\allfiles\undefined
\documentclass[12pt, a4paper,oneside, UTF8]{ctexbook}
\usepackage[dvipsnames]{xcolor}
\usepackage{mathtools}   % 数学公式
\usepackage{amsthm}    % 定理环境
\usepackage{amssymb}   % 更多公式符号
\usepackage{graphicx}  % 插图
%\usepackage{mathrsfs}  % 数学字体
%\usepackage{newtxtext,newtxmath}
%\usepackage{arev}
\usepackage{kmath,kerkis}
\usepackage{newtxtext}
\usepackage{bbm}
\usepackage{enumitem}  % 列表
\usepackage{geometry}  % 页面调整
%\usepackage{unicode-math}
\usepackage[colorlinks,linkcolor=black]{hyperref}

\usepackage{wrapfig}


\usepackage{ulem}	   % 用于更多的下划线格式,
					   % \uline{}下划线,\uuline{}双下划线,\uwave{}下划波浪线,\sout{}中间删除线,\xout{}斜删除线
					   % \dashuline{}下划虚线,\dotuline{}文字底部加点


\graphicspath{ {flg/},{../flg/}, {config/}, {../config/} }  % 配置图形文件检索目录
\linespread{1.5} % 行高

% 页码设置
\geometry{top=25.4mm,bottom=25.4mm,left=20mm,right=20mm,headheight=2.17cm,headsep=4mm,footskip=12mm}

% 设置列表环境的上下间距
\setenumerate[1]{itemsep=5pt,partopsep=0pt,parsep=\parskip,topsep=5pt}
\setitemize[1]{itemsep=5pt,partopsep=0pt,parsep=\parskip,topsep=5pt}
\setdescription{itemsep=5pt,partopsep=0pt,parsep=\parskip,topsep=5pt}

% 定理环境
% ########## 定理环境 start ####################################
\theoremstyle{definition}
\newtheorem{defn}{\indent 定义}[section]

\newtheorem{lemma}{\indent 引理}[section]    % 引理 定理 推论 准则 共用一个编号计数
\newtheorem{thm}[lemma]{\indent 定理}
\newtheorem{corollary}[lemma]{\indent 推论}
\newtheorem{criterion}[lemma]{\indent 准则}

\newtheorem{proposition}{\indent 命题}[section]
\newtheorem{example}{\indent \color{SeaGreen}{例}}[section] % 绿色文字的 例 ,不需要就去除\color{SeaGreen}{}
\newtheorem*{rmk}{\indent \color{red}{注}}

% 两种方式定义中文的 证明 和 解 的环境:
% 缺点:\qedhere 命令将会失效【技术有限,暂时无法解决】
\renewenvironment{proof}{\par\textbf{证明.}\;}{\qed\par}
\newenvironment{solution}{\par{\textbf{解.}}\;}{\qed\par}

% 缺点:\bf 是过时命令,可以用 textb f等替代,但编译会有关于字体的警告,不过不影响使用【技术有限,暂时无法解决】
%\renewcommand{\proofname}{\indent\bf 证明}
%\newenvironment{solution}{\begin{proof}[\indent\bf 解]}{\end{proof}}
% ######### 定理环境 end  #####################################

% ↓↓↓↓↓↓↓↓↓↓↓↓↓↓↓↓↓ 以下是自定义的命令  ↓↓↓↓↓↓↓↓↓↓↓↓↓↓↓↓

% 用于调整表格的高度  使用 \hline\xrowht{25pt}
\newcommand{\xrowht}[2][0]{\addstackgap[.5\dimexpr#2\relax]{\vphantom{#1}}}

% 表格环境内长内容换行
\newcommand{\tabincell}[2]{\begin{tabular}{@{}#1@{}}#2\end{tabular}}

% 使用\linespread{1.5} 之后 cases 环境的行高也会改变,重新定义一个 ca 环境可以自动控制 cases 环境行高
\newenvironment{ca}[1][1]{\linespread{#1} \selectfont \begin{cases}}{\end{cases}}
% 和上面一样
\newenvironment{vx}[1][1]{\linespread{#1} \selectfont \begin{vmatrix}}{\end{vmatrix}}

\def\d{\textup{d}} % 直立体 d 用于微分符号 dx
\def\R{\mathbb{R}} % 实数域
\def\N{\mathbb{N}} % 自然数域
\def\C{\mathbb{C}} % 复数域
\def\Z{\mathbb{Z}} % 整数环
\def\Q{\mathbb{Q}} % 有理数域
\newcommand{\bs}[1]{\boldsymbol{#1}}    % 加粗,常用于向量
\newcommand{\ora}[1]{\overrightarrow{#1}} % 向量

% 数学 平行 符号
\newcommand{\pll}{\kern 0.56em/\kern -0.8em /\kern 0.56em}

% 用于空行\myspace{1} 表示空一行 填 2 表示空两行  
\newcommand{\myspace}[1]{\par\vspace{#1\baselineskip}}

%s.t. 用\st就能打出s.t.
\DeclareMathOperator{\st}{s.t.}

%罗马数字 \rmnum{}是小写罗马数字, \Rmnum{}是大写罗马数字
\makeatletter
\newcommand{\rmnum}[1]{\romannumeral #1}
\newcommand{\Rmnum}[1]{\expandafter@slowromancap\romannumeral #1@}
\makeatother
\begin{document}
	% \title{{\Huge{\textbf{$Functional \,\, Analysis$}}}\footnote{参考书籍:\\
			\hspace*{4em} \textbf{《Linear and Nonlinear Functional Analysis with Applications》 -- Philippe G. Ciarlet} \\
			\hspace*{4em} \textbf{《Real  Analysis -- Modern Techniques and Their Applications》 -- Gerald  B.  Folland} \\
			\hspace*{4em} \textbf{《Functional Analysis -- Introduction to Further Topics in Analysis》 -- Elias M. Stein} \\
			\hspace*{4em} \textbf{《泛函分析讲义》 -- 张恭庆、林源渠} 
			}}
\author{$-TW-$}
\date{\today}
\maketitle                   % 在单独的标题页上生成一个标题

\thispagestyle{empty}        % 前言页面不使用页码
\begin{center}
	\Huge\textbf{序}
\end{center}


\vspace*{3em}
\begin{center}
	\large{\textbf{天道几何,万品流形先自守;}}\\
	
	\large{\textbf{变分无限,孤心测度有同伦。}}
\end{center}

\vspace*{3em}
\begin{flushright}
	\begin{tabular}{c}
		\today \\ \small{\textbf{长夜伴浪破晓梦,梦晓破浪伴夜长}}
	\end{tabular}
\end{flushright}


\newpage                      % 新的一页
\pagestyle{plain}             % 设置页眉和页脚的排版方式(plain:页眉是空的,页脚只包含一个居中的页码)
\setcounter{page}{1}          % 重新定义页码从第一页开始
\pagenumbering{Roman}         % 使用大写的罗马数字作为页码
\tableofcontents              % 生成目录

\newpage                      % 以下是正文
\pagestyle{plain}
\setcounter{page}{1}          % 使用阿拉伯数字作为页码
\pagenumbering{arabic}
\setcounter{chapter}{0}    % 设置 -1 可作为第零章绪论从第零章开始 
	\else
	\fi
	%  ############################ 正文部分
\chapter{赋范空间}
\section{赋范线性空间}
\subsection{赋范线性空间}
	这一章我们将来讨论\textbf{赋范线性空间}的相关定义及性质. 首先回顾\textbf{范数}的定义\textbf{(同定义 \ref{def 1.1.1})}.
	
	\begin{defn}\label{def 2.1.1}
		Let $X$ be a vector space over field $\mathbb{F}$, a \underline{\textcolor{blue}{\textbf{norm}}} is a function:
		\begin{align}
			X &\longrightarrow \R_{\geq 0} \\
			f &\longmapsto \Vert f \Vert
		\end{align}
		satisfying the following properties:
		\begin{enumerate}
			\item[(\rmnum{1})]$\Vert f \Vert \geq 0$, $\forall f \in X$. \hspace*{3em} ($\Vert f \Vert = 0 \,\, \Leftrightarrow \,\, f = 0$)
			
			\item[(\rmnum{2})]$\Vert af \Vert = \left| a \right| \Vert f \Vert$, $\forall a \in \mathbb{F}, f \in X$.
			
			\item[(\rmnum{3})]$\Vert f + g \Vert \leq \Vert f \Vert + \Vert g \Vert$, $\forall f , g \in X$.
		\end{enumerate}
		
		\vspace*{0.5em}
		
		\begin{rmk}
			\begin{itemize}
				\item 事实上\textbf{赋范线性空间}的称呼有些许不妥, 应直接称\textbf{赋范空间}. 因为\textbf{范数}的概念本就需要在\textbf{线性空间}上定义. 
				\begin{center}
					(否则\textbf{范数定义第(\rmnum{3})条“三角不等式”}中“$f + g$” 的加法无从定义)
				\end{center}
				
				\vspace*{2em}
				
				\item \textbf{赋范空间}与\textbf{度量空间}的关系为:
				\begin{center}
					\textbf{赋范空间} $\,\, \Rightarrow \,\,$ \textbf{度量空间} 
					\hspace*{1em} , \hspace*{1em} 
					\textbf{度量空间} $\,\, \not\Rightarrow \,\,$ \textbf{赋范空间}
				\end{center}
				即\textbf{赋范空间}上均可定义度量, 但\textbf{度量空间}却不一定能扩充为\textbf{赋范空间}. \\ 其中对于任意\textbf{赋范空间}$(X , \Vert \cdot \Vert)$, 其上总是默认定义如下\textbf{度量$d$}:
				\[ d(x , y) = \Vert x - y \Vert , \,\, \forall x , y \in X \]
				不难证明其满足\textbf{度量}的三条公理 (\textbf{Def \ref{def 1.1.2}})
				
				\newpage
				
				\item 对于“\textbf{度量空间} $\,\, \not\Rightarrow \,\,$ \textbf{赋范空间}”, 先来明确\textbf{度量空间$(X , \rho)$}要扩充为\textbf{赋范空间}所需条件:
				
				\vspace*{1em}
				
				\begin{enumerate}
					\item 引入(数)域$\mathbb{F}$, 在$X$ 中定义加法$\&$ 数乘运算, 使其满足线性空间八大公理. 即度量$\rho$ 需要先定义在线性空间上, 使其称为\textbf{度量线性空间}. 
					
					\vspace*{0.5em}
					
					\item 度量$\rho$ 需要满足\textbf{平移不变性}, 即
					\[ \rho(x , 0) = \rho(x + y , y) , \,\, \forall x , y \in X \]
					
					\item 对应\textbf{范数}定义的\textbf{绝对齐性}, 度量$\rho$ 也需要满足
					\[ \rho(\alpha x , 0) = \left| \alpha \right| \rho(x , 0) , \,\, \forall x \in X , \alpha \in \mathbb{F} \]
				\end{enumerate}
				
				\vspace*{1em}
				
				而即便是对于\textbf{度量线性空间}, 大多数也并不满足后两者条件, 下面给出一个反例.
				
				\begin{example}\label{ex 2.1.1}
					在欧氏空间$\R$ 中定义度量$\rho$:
					\[ \rho(x , y) = \frac{\left| x - y \right|}{1 + \left| x - y \right|} , \,\, \forall x , y \in \R \]
					则$\rho$ 不满足绝对齐性, 从而无法扩充为\textbf{赋范空间}.
				\end{example}
				
				\vspace*{6em}
				
				\item 对于任意\textbf{赋范空间}$(X , \Vert \cdot \Vert)$, 其上定义的范数$\Vert \cdot \Vert$ 均为连续函数.
				
				\begin{itemize}
					\item 
					\begin{proof}
						下面分两步进行证明, 首先证明一条引理. 
						\begin{enumerate}
							\item $\forall x , y \in X$, $\Big| \, \Vert x \Vert - \Vert y \Vert \, \Big| \leq \Vert x - y \Vert$:\\
							根据范数的三角不等式, $\forall x , y \in X$,
							\[ \Vert x \Vert \leq \Vert x - y \Vert + \Vert y \Vert \]
							\[ \Vert y \Vert \leq \Vert y - x \Vert + \Vert x \Vert = \Vert x - y \Vert \]
							移项后可得:$\Big| \, \Vert x \Vert - \Vert y \Vert \, \Big| \leq \Vert x - y \Vert$.
							
							\vspace*{2em}
							
							\item $\Vert \cdot \Vert$ 连续:$\forall \{ x_n \}_{n = 1}^{\infty} \subset X$ with $x_n \to x \in X$. Since
							\[ \Big| \, \Vert x_n \Vert - \Vert x \Vert \, \Big| \leq \Vert x_n - x \Vert \]
							Thus $\Vert x_n \Vert \to \Vert x \Vert$, 即$\Vert \cdot \Vert$ 连续.
						\end{enumerate}
					\end{proof}
				\end{itemize}
			\end{itemize}
		\end{rmk}
	\end{defn}

\newpage

\subsection{Banach空间}
	下面给出\textbf{Banach空间}的定义.
	
	\begin{defn}\label{def 2.1.2}
		完备的$B^*$ 空间称为\underline{\textcolor{blue}{\textbf{$B$ 空间}}} / \underline{\textcolor{blue}{\textbf{Banach空间}}}. 
		
		\vspace*{2em}
		
		\begin{rmk}
			\begin{itemize}
				\item 我们称\textbf{赋范空间}为\underline{\textcolor{blue}{\textbf{$B^*$ 空间}}}. 
				
				\vspace*{1em}
				
				\item 若$X$ 为\textbf{赋范空间}, 我们记$X \in B^*$;若$X$ 为\textbf{Banach空间}, 我们记$X \in B$.
			\end{itemize}
		\end{rmk}
	\end{defn}






	%  ############################
	\ifx\allfiles\undefined
\end{document}
\fi