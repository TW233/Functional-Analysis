\ifx\allfiles\undefined
\input{../config/config}
\begin{document}
	% \input{../config/cover} 
	\else
	\fi
	%  ############################ 正文部分
\chapter{赋范空间}
\section{赋范线性空间}
\subsection{赋范线性空间}
	这一章我们将来讨论\textbf{赋范线性空间}的相关定义及性质. 首先回顾\textbf{范数}的定义\textbf{(同定义 \ref{def 1.1.1})}.
	
	\begin{defn}\label{def 2.1.1}
		Let $X$ be a vector space over field $\mathbb{F}$, a \underline{\textcolor{blue}{\textbf{norm}}} is a function:
		\begin{align}
			X &\longrightarrow \R_{\geq 0} \\
			f &\longmapsto \Vert f \Vert
		\end{align}
		satisfying the following properties:
		\begin{enumerate}
			\item[(\rmnum{1})]$\Vert f \Vert \geq 0$, $\forall f \in X$. \hspace*{3em} ($\Vert f \Vert = 0 \,\, \Leftrightarrow \,\, f = 0$)
			
			\item[(\rmnum{2})]$\Vert af \Vert = \left| a \right| \Vert f \Vert$, $\forall a \in \mathbb{F}, f \in X$.
			
			\item[(\rmnum{3})]$\Vert f + g \Vert \leq \Vert f \Vert + \Vert g \Vert$, $\forall f , g \in X$.
		\end{enumerate}
		
		\vspace*{0.5em}
		
		\begin{rmk}
			\begin{itemize}
				\item 事实上\textbf{赋范线性空间}的称呼有些许不妥, 应直接称\textbf{赋范空间}. 因为\textbf{范数}的概念本就需要在\textbf{线性空间}上定义. 
				\begin{center}
					(否则\textbf{范数定义第(\rmnum{3})条“三角不等式”}中“$f + g$” 的加法无从定义)
				\end{center}
				
				\vspace*{2em}
				
				\item \textbf{赋范空间}与\textbf{度量空间}的关系为:
				\begin{center}
					\textbf{赋范空间} $\,\, \Rightarrow \,\,$ \textbf{度量空间} 
					\hspace*{1em} , \hspace*{1em} 
					\textbf{度量空间} $\,\, \not\Rightarrow \,\,$ \textbf{赋范空间}
				\end{center}
				即\textbf{赋范空间}上均可定义度量, 但\textbf{度量空间}却不一定能扩充为\textbf{赋范空间}. \\ 其中对于任意\textbf{赋范空间}$(X , \Vert \cdot \Vert)$, 其上总是默认定义如下\textbf{度量$d$}:
				\[ d(x , y) = \Vert x - y \Vert , \,\, \forall x , y \in X \]
				不难证明其满足\textbf{度量}的三条公理 (\textbf{Def \ref{def 1.1.2}})
				
				\newpage
				
				\item 对于“\textbf{度量空间} $\,\, \not\Rightarrow \,\,$ \textbf{赋范空间}”, 先来明确\textbf{度量空间$(X , \rho)$}要扩充为\textbf{赋范空间}所需条件:
				
				\vspace*{1em}
				
				\begin{enumerate}
					\item 引入(数)域$\mathbb{F}$, 在$X$ 中定义加法$\&$ 数乘运算, 使其满足线性空间八大公理. 即度量$\rho$ 需要先定义在线性空间上, 使其称为\textbf{度量线性空间}. 
					
					\vspace*{0.5em}
					
					\item 度量$\rho$ 需要满足\textbf{平移不变性}, 即
					\[ \rho(x , 0) = \rho(x + y , y) , \,\, \forall x , y \in X \]
					
					\item 对应\textbf{范数}定义的\textbf{绝对齐性}, 度量$\rho$ 也需要满足
					\[ \rho(\alpha x , 0) = \left| \alpha \right| \rho(x , 0) , \,\, \forall x \in X , \alpha \in \mathbb{F} \]
				\end{enumerate}
				
				\vspace*{1em}
				
				而即便是对于\textbf{度量线性空间}, 大多数也并不满足后两者条件, 下面给出一个反例.
				
				\begin{example}\label{ex 2.1.1}
					在欧氏空间$\R$ 中定义度量$\rho$:
					\[ \rho(x , y) = \frac{\left| x - y \right|}{1 + \left| x - y \right|} , \,\, \forall x , y \in \R \]
					则$\rho$ 不满足绝对齐性, 从而无法扩充为\textbf{赋范空间}.
				\end{example}
				
				\vspace*{6em}
				
				\item 对于任意\textbf{赋范空间}$(X , \Vert \cdot \Vert)$, 其上定义的范数$\Vert \cdot \Vert$ 均为连续函数.
				
				\begin{itemize}
					\item 
					\begin{proof}
						下面分两步进行证明, 首先证明一条引理. 
						\begin{enumerate}
							\item $\forall x , y \in X$, $\Big| \, \Vert x \Vert - \Vert y \Vert \, \Big| \leq \Vert x - y \Vert$:\\
							根据范数的三角不等式, $\forall x , y \in X$,
							\[ \Vert x \Vert \leq \Vert x - y \Vert + \Vert y \Vert \]
							\[ \Vert y \Vert \leq \Vert y - x \Vert + \Vert x \Vert = \Vert x - y \Vert \]
							移项后可得:$\Big| \, \Vert x \Vert - \Vert y \Vert \, \Big| \leq \Vert x - y \Vert$.
							
							\vspace*{2em}
							
							\item $\Vert \cdot \Vert$ 连续:$\forall \{ x_n \}_{n = 1}^{\infty} \subset X$ with $x_n \to x \in X$. Since
							\[ \Big| \, \Vert x_n \Vert - \Vert x \Vert \, \Big| \leq \Vert x_n - x \Vert \]
							Thus $\Vert x_n \Vert \to \Vert x \Vert$, 即$\Vert \cdot \Vert$ 连续.
						\end{enumerate}
					\end{proof}
				\end{itemize}
			\end{itemize}
		\end{rmk}
	\end{defn}

\newpage

\subsection{Banach空间}
	下面给出\textbf{Banach空间}的定义.
	
	\begin{defn}\label{def 2.1.2}
		完备的$B^*$ 空间称为\underline{\textcolor{blue}{\textbf{$B$ 空间}}} / \underline{\textcolor{blue}{\textbf{Banach空间}}}. 
		
		\vspace*{2em}
		
		\begin{rmk}
			\begin{itemize}
				\item 我们称\textbf{赋范空间}为\underline{\textcolor{blue}{\textbf{$B^*$ 空间}}}. 
				
				\vspace*{1em}
				
				\item 若$X$ 为\textbf{赋范空间}, 我们记$X \in B^*$;若$X$ 为\textbf{Banach空间}, 我们记$X \in B$.
			\end{itemize}
		\end{rmk}
	\end{defn}
	
	\vspace{4em}
	
	\textbf{Banach空间}事实上十分常见, 下面给出两个经典的例子. 首先便是连续函数构成的空间.
	\begin{example}\label{ex 2.1.2}
		\textbf{[Banach空间]}. 
		$(C[a , b] , \Vert \cdot \Vert_{\infty})$ 即为Banach空间, 其中$\Vert \cdot \Vert_{\infty} = \underset{[a , b]}{\max} \left| \cdot \right|$. 
			
		\vspace{1em}
			
		\begin{proof}
			根据\textbf{命题 \ref{prop 1.7.1}}即可得证. 
		\end{proof}
	\end{example}
	
	
	\vspace{6em}
	下面我们来证明实分析中的$L^p$ 空间为Banach空间, 即实分析中的\textbf{Riesz-Fisher定理\footnote{详情可见\textbf{Real Analysis -- Folland , P183 Theorem 6.6}.}}. 
	\begin{thm}\label{thm 2.1.1}
		\textbf{[Riesz-Fisher定理]}. 
		\begin{center}
			$(L^{p}[a , b] , \Vert \cdot \Vert_p)$ 为Banach空间, 其中$\Vert \cdot \Vert_p = \left( \int_{[a , b]} \left| \cdot \right|^p \right)^{\tfrac{1}{p}}$. 
		\end{center}
		
		\vspace{6em}
		
		\begin{proof}
			根据\textbf{Minkowski不等式 (定理 \ref{thm 1.1.4})}, 不难证明$(L^{p}[a,  b] , \Vert \cdot \Vert_{p})$ 为赋范空间. \\
			下面证明其完备性:\\
			$\forall$ Cauchy sequence $\{ f_n \}_{n = 1}^{\infty} \subset L^{p}[a , b]$. i.e. $\forall \epsilon > 0$, $\exists N_\epsilon \in \N$, $\st$
			\[ \Vert f_m - f_n \Vert_{p} \leq \epsilon , \,\, \forall m , n \geq N_\epsilon \]
			For $\epsilon = \dfrac{1}{2}$, $\exists n_1 \in \N$, $\st$
			\[ \Vert f_m - f_n \Vert_{p} \leq \frac{1}{2} , \,\, \forall m , n \geq n_1 \]
			For $\epsilon = \dfrac{1}{4}$, $\exists n_2 > n_1$, $\st$
			\[ \Vert f_m - f_n \Vert_{p} \leq \frac{1}{4} , \,\, \forall m , n \geq n_2 \]
			\begin{center}
				$\cdots$
			\end{center}
			Then we get a subsequence $\{ f_{n_k} \}_{k = 1}^{\infty} \subset \{ f_n \}_{n = 1}^{\infty}$, $\st$
			\[ \Vert f_{n_k} - f_{n_{k + 1}} \Vert_p \leq \frac{1}{2^k} , \,\, \forall k \in \N \]
			Since $\{ f_n \}_{n = 1}^{\infty} \subset L^{p}[a , b]$ is a Cauchy sequence, thus \\
			要证:$\{ f_n \}_{n = 1}^{\infty}$ 收敛. \\
			只需证:$\{ f_{n_k} \}_{k = 1}^{\infty} \subset \{ f_n \}_{n = 1}^{\infty}$ 收敛. Let
			\[ g_{m}(x) = \sum_{k = 1}^{m} \left| f_{n_{k + 1}} - f_{n_k} \right| (x) , \,\, \forall m \in \N , x \in [a , b] \]
			Then $g_m \in L^p$, $g_m \geq 0$ and $g_{m} \leq g_{m + 1}$ 单调递增, and
			\[ g_{m}(x) \leq \sum_{k = 1}^{m} \frac{1}{2^k} \leq 1 , \,\, \forall m \in \N \]
			记
			\[ g(x) = \lim_{m \to \infty} g_{m}(x) = \sum_{k = 1}^{\infty} \left| f_{n_{k + 1}} - f_{n_k} \right| (x) \leq 1 \]
			Then $g_{m}^p \in L^1$, $g_{m}^p \geq 0$, $g_{m}^p \leq g_{m + 1}^p$. By \textbf{MCT (单调收敛定理)\footnote{\textbf{Monotone Convergence Theorem}, 详见$Real \,\, Analysis$ 笔记 \textbf{定理 3.1.2 $\&$ 定理 3.1.4}.}}, 
			\[ \lim_{m \to \infty} \int_{[a , b]} g_{m}^p(x) \, d\mu = \int_{[a , b]} \lim_{m \to \infty} g_{m}^p(x) \, d\mu \]
			Thus
			\begin{align}
				\left( \int_{[a , b]} \left| g(x) \right|^p \, d\mu \right)^{\tfrac{1}{p}} 
				&= \lim_{m \to \infty} \left( \int_{[a , b]} \left| g_{m}(x) \right|^p \, d\mu \right)^{\tfrac{1}{p}} \\
				&\leq \lim_{m \to \infty} \left( \int_{[a , b]} 1 \, d\mu \right)^{\tfrac{1}{p}} \\
				&= (b - a)^{\tfrac{1}{p}} < \infty
			\end{align}
			i.e. $\Vert g \Vert_{p} \leq (b - a)^{\tfrac{1}{p}} < \infty$, then $g \in L^p$. 故$g_m \overset{a.e.}{\to} g \in L^p$. \\
			由于$\overset{m}{\underset{k = 1}{\sum}} \left| f_{n_{k + 1}} - f_{n_k} \right|(x)$ 与$f_{n_k}(x)$ 有相同的收敛性, 因此$\exists f \in L^p$, $\st$
			\[ f_{n_k} \overset{a.e.}{\to} f \,\, as \,\, k \to \infty \]
			下面来证明$f_{n_k}$ 在$p$-范数$\Vert \cdot \Vert_p$ 意义下收敛于$f$:\\
			考虑函数列$\{ \left| f_{n_k} - f \right| \}_{k = 1}^{\infty} \subset L^p$. Since
			\begin{align}
				\left| f_{n_k} - f \right| 
				= \sum_{j = k}^{\infty} \left| f_{n_{j + 1}} - f_{n_j} \right| 
				= g - g_{k - 1} \leq 2g
			\end{align}
			Thus 函数列$\{ \left| f_{n_k} - f \right| \}_{k = 1}^{\infty}$ 被可积函数$2g \in L^p$ 所控制. \\
			$\Rightarrow \,\, \{ \left| f_{n_k} - f \right|^p \}_{k = 1}^{\infty} \subset L^1$ 可被$(2g)^p \in L^1$ 所控制. 根据\textbf{DCT (控制收敛定理)\footnote{\textbf{Dominated Convergence Theorem}, 详见$Real \,\, Analysis$ 笔记 \textbf{Thm 3.1.7}.}}, 
			\begin{align}
				\lim_{k \to \infty} \int_{[a , b]} \left| f_{n_k} - f \right|^p \, d\mu 
				= \int_{[a , b]} \lim_{k \to \infty} \left| f_{n_k} - f \right|^p \, d\mu = 0
			\end{align}
			i.e. $\Vert f_{n_k} - f \Vert_{p} \to 0$ as $k \to \infty$. 故$\{ f_{n_k} \}_{k = 1}^{\infty}$ 收敛, 从而$\{ f_n \}_{n = 1}^{\infty}$ 收敛, $L^p$ complete.
		\end{proof}
	\end{thm}






	%  ############################
	\ifx\allfiles\undefined
\end{document}
\fi