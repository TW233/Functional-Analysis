\ifx\allfiles\undefined
\input{../config/config}
\begin{document}
	% \input{../config/cover} 
	\else
	\fi
	%  ############################ 正文部分
\chapter{线性算子与线性泛函}
	
\section{线性算子$\&$ 线性泛函}
	这一节我们主要来给出\textbf{线性算子}及\textbf{线性泛函}相关概念的定义. \textbf{线性算子}为高等代数中学过的\textbf{线性变换 (线性映射)}的推广, 即更多的在\textbf{无穷维线性空间}上进行讨论. 而\textbf{线性泛函}是\textbf{线性算子}的一个特例, 即将线性泛函的陪域取作数域, 相当于定义域较大的函数, 即为函数的推广. 
	
	\vspace{1em}
	
	\begin{defn}\label{def 4.1.1}
		设$X , Y$ 为定义在数域$\mathbb{K}$ 上的线性空间. 若映射$T : X \longrightarrow Y$ 为线性映射, 即
		\[ T(\alpha x + \beta y) = \alpha T(x) + \beta T(y) , \,\, \forall x , y \in X , \,\, \forall \alpha , \beta \in \mathbb{K} \]
		则称$T$ 为$X$ 到$Y$ 上的一个\underline{\textcolor{blue}{\textbf{线性算子}}}. 特别地, 若$Y = \mathbb{K} \in \{ \R , \C \}$, 则称$T$ 为\underline{\textcolor{blue}{\textbf{线性泛函}}}. 
		
		\vspace{4em}
		
		\begin{rmk}
			\begin{itemize}
				\item 此处我们沿用高代与范畴论中的记号, 将从线性空间$X$ 到$Y$ 上的所有\textbf{线性算子}构成的空间记作\textbf{$Hom(X , Y)$}. 不难说明$Hom(X , Y)$ 也是个定义在数域$\mathbb{K}$ 上的\textbf{线性空间}. 
				
				\vspace*{4em}
				
				\item 回顾高代中有关\textbf{有限维线性空间的对偶空间}的结论:
				\begin{center}
					\textbf{数域$\mathbb{K}$ 上有限维线性空间的所有线性泛函构成空间 (对偶空间)同构于$\mathbb{K}^n$}. 
				\end{center} 
				
				\vspace*{2em}
				
				\begin{proof}
					设$X$ 为数域$\mathbb{K}$ 上$n$ 维线性空间, $\{ e_i \}_{i = 1}^n \subset X$ 为一组基. \\
					Then for $\forall f \in Hom(X , \mathbb{K})$, 
					\[ f(x) = f \left( \sum_{i = 1}^n x_i e_i \right) = \sum_{i = 1}^n x_i f(e_i) , \,\, \forall x = \sum_{i = 1}^n x_i e_i \in X \]
					Consider the mapping 
					\begin{align}
						T : Hom(X , \mathbb{K}) &\longrightarrow \mathbb{K}^n \\
						f &\longmapsto \Big( f(e_1) , \,\, \cdots , \,\, f(e_n) \Big)
					\end{align}
					It's not hard to prove that $T$ is an isomorphism between $Hom(X , \mathbb{K})$ and $\mathbb{K}^n$, i.e. 
					\[ Hom(X , \mathbb{K}) \cong \mathbb{K}^n \]
				\end{proof}
			\end{itemize}
		\end{rmk}
	\end{defn}
	
	\vspace*{6em}
	
	对于线性空间$X$ 到$Y$ 上的线性算子$Hom(X , Y)$, 为了考虑其连续性, 下面将$X , Y$ 限制为$B^*$ 空间, 给出\textbf{连续算子}的定义. 
	
	\vspace*{1em}
	
	\begin{defn}\label{def 4.1.2}
		设$X , Y \in B^{*}$, $T \in Hom(X , Y)$. 若$T$ 连续, 即在$X$ 中每点处连续, 则称$T$ 为\underline{\textcolor{blue}{\textbf{连续算子}}}, 记$X$ 到$Y$ 的连续算子全体为\underline{\textcolor{blue}{\textbf{$\, L(X , Y)$}}}. 特别地, 若$Y = \mathbb{K} \in \{ \R , \C \}$, 记\underline{\textcolor{blue}{\textbf{$\, X^* = L(X , \mathbb{K})$}}}. 
		
		\vspace*{4em}
		
		\begin{rmk}
			\begin{itemize}
				\item 事实上, 此处记号$X^*$ 是对高代中\textbf{对偶空间}概念的推广, 即\uwave{当$X$ 为\textbf{有限维}线性空间时, $X^* = L(X , \mathbb{K})$ 就是$X$ 上的对偶空间}, 故更多讨论无穷维的情况. 这一点由下述命题保证:
				\begin{center}
					\textbf{有限维$B^*$ 空间$(X , \Vert \cdot \Vert)$ 上的线性函数$f : X \longrightarrow \mathbb{K}$ 连续}. 
				\end{center}
				
				\vspace*{1em}
				
				\begin{proof}
					根据\textbf{有限维$B^*$ 空间范数的等价性 (Thm \ref{thm 2.2.2})} 容易证明, 于是$Hom(X , \mathbb{K}) = X^*$.  
				\end{proof}
				
				\vspace*{6em}
				
				\item 对于$\forall T \in Hom(X , \mathbb{K})$, 根据$T$ 的线性性, 不难得到$T \in L(X , Y) \,\, \Leftrightarrow \,\, T$ 在$x = 0$ 处连续. 
			\end{itemize}
		\end{rmk}
	\end{defn}
	
	\newpage
	
	下面再给出\textbf{有界算子}的概念. 
	
	\vspace*{1em}
	
	\begin{defn}\label{def 4.1.3}
		设$X , Y \in B^*$ 且$T \in Hom(X , Y)$. 如果$T$ 将任何有界集映为有界集, 则称$T$ 为\underline{\textcolor{blue}{\textbf{有界 (线性)算子}}}. 
		
		\vspace*{4em}
		
		\begin{rmk}
			此处给出$T$ 有界的几个\textbf{等价定义}, 即对于$\forall T \in Hom(X , Y)$, 
			\begin{center}
				\textbf{$T$ 有界 $\,\, \Leftrightarrow \,\, $ 单位球 (面)的像有界 $\,\, \Leftrightarrow \,\, \exists M > 0 , \,\, \st \Vert T x \Vert_Y \leq M \, \Vert x \Vert_X , \,\, \forall x \in X$}
			\end{center}
			
			\vspace*{2em}
			
			\begin{proof}
				\begin{itemize}
					\item $T$ 有界 $\,\, \Leftrightarrow \,\, $ 单位球 (面)的像有界:必要性显然. 下面证充分性$\Leftarrow$:$\exists M > 0$, $\st$
					\[ \Vert Tx \Vert_Y \leq M \, \Vert x \Vert_X = M , \,\, \forall x \in B(0 , 1) \subset X \]
					$\forall A \subset X$ bounded, i.e. $\exists r > 0$, $\st A \subset B(0 , r)$. Then by the linearity of $T \in Hom(X , Y)$, 
					\[ \Vert T x \Vert_Y 
					= \Big\Vert T \Big( \Vert x \Vert_X \cdot \frac{x}{\Vert x \Vert_X} \Big) \Big\Vert_Y 
					= \Vert x \Vert_X \cdot \Big\Vert T \Big( \frac{x}{\Vert x \Vert_X} \Big) \Big\Vert_Y 
					\leq r \cdot M < \infty , \,\, \forall x \in A \]
					Therefore, $T(A) \subset \mathbb{K}$ bounded. $T$ bounded. 
					
					\vspace*{8em}
					
					\item 单位球 (面)的像有界 $\,\, \Leftrightarrow \,\, \exists M > 0 , \,\, \st \Vert T x \Vert_Y \leq M \, \Vert x \Vert_X , \,\, \forall x \in X$:\\
					充分性显然. 下面证明必要性$\Rightarrow$:$\exists M > 0$, $\st$
					\[ \Vert Tx \Vert_Y \leq M , \,\, \forall x \in \partial B(0 , 1) \]
					Then for $\forall x \in X$, 
					\[ \Big\Vert T\Big( \frac{x}{\Vert x \Vert_X} \Big) \Big\Vert_Y 
					= \frac{\Vert Tx \Vert_Y}{\Vert x \Vert_X}
					\leq M , \,\, \forall x \in M \]
					i.e. $\Vert Tx \Vert_Y \leq M \, \Vert x \Vert_X , \,\, \forall x \in X$.
				\end{itemize}
			\end{proof}
		\end{rmk}
	\end{defn}
	
	\newpage
	
	下面来介绍有关线性算子的十分重要的结论, 即对于$B^*$ 空间上的线性算子, \textbf{连续与有界等价}. 
	
	\begin{proposition}\label{prop 4.1.1}
		\textbf{[$B^*$ 空间线性算子连续$\Leftrightarrow$ 有界]}. \\
		Suppose $X , Y \in B^*$, $T \in Hom(X , Y)$, then
		\begin{center}
			\textbf{T连续 $\,\, \Leftrightarrow \,\, $T 有界}
		\end{center}
		
		\vspace*{4em}
		
		\begin{proof}
			\begin{itemize}
				\item \textbf{充分性$\Leftarrow$}:Suppose $T \in Hom(X, Y)$ bounded, then \\
				By \textbf{$B^*$ 空间上有界线性算子等价定义 (Def \ref{def 4.1.3})}, $\exists M > 0$, $\st$
				\[ \Vert Tx \Vert_Y \leq M \, \Vert x \Vert_X , \,\, \forall x \in X \]
				Thus $T : X \longrightarrow Y$ is lipschitz continuous, specifically continuous. 
				
				\vspace*{6em}
				
				\item \textbf{必要性$\Rightarrow$}:Suppose $T \in Hom(X , Y)$ continuous. \\
				反证法. 假设$T$ 无界, 则$\forall n \in \N$, $\exists x_n \in X$, $\st$
				\[ \Vert T x_n \Vert_Y > n \, \Vert x_n \Vert_X , \,\, \forall n \in \N \]
				i.e. 
				\[ \left\Vert T \left( \frac{1}{n} \cdot \frac{x_n}{\Vert x_n \Vert_X} \right) \right\Vert_Y > 1 , \,\, \forall n \in \N \]
				Let $y_n = \dfrac{x_n}{n \Vert x_n \Vert_X} \in X$, then letting $n \to \infty$, we have
				\[ \Vert y_n \Vert_X \to 0 , \,\, \Vert Ty_n \Vert_Y > 1 \not\to 0 \,\, \text{as} \,\, n \to \infty \]
				Therefore, $T(x)$ is discontinuous at $x = \overrightarrow{0}$. 从而$T$ 不连续, 矛盾.
			\end{itemize}
		\end{proof}
	\end{proposition}
	
	\newpage
	
	下面给出$B^*$ 空间上\textbf{线性泛函连续的充要条件}, 即其\textbf{核空间为闭集}. 
	
	\begin{proposition}\label{prop 4.1.2}
		\textbf{[$B^*$ 空间上线性泛函连续的充要条件]}. \\
		Suppose $X \in B^*$, $f \in Hom(X , \mathbb{K})$. Then
		\begin{center}
			\textbf{$f \in X^* \,\, \Leftrightarrow \,\, Ker f \subset X$ closed}
		\end{center}
		
		\vspace*{2em}
		
		\begin{rmk}
			回顾线性空间$X , Y$ 之间线性映射$f \in Hom(X , Y)$ 的核空间的定义
			\[ Kerf = \{ x \in X \mid f(x) = 0 \in Y \} \subset X \]
			此命题即说明了$B^*$ 空间上线性泛函连续$\,\, \Leftrightarrow \,\,$ 其核空间为闭集. 
		\end{rmk}
		
		\vspace*{4em}
		
		\begin{proof}
			\begin{itemize}
				\item \textbf{必要性$\Rightarrow$}:Suppose $f \in X^* = L(X , Y)$ continuous. Then \\
				For $\forall \{ x_n \}_{n = 1}^{\infty} \subset Ker f$ with $x_n \overset{\Vert \cdot \Vert_X}{\to} x \in X$ converges, since $f$ is continuous, then
				\[ f(x) = f \left( \lim_{n \to \infty} x_n \right) = \lim_{n \to \infty} f(x_n) = 0 \]
				Thus $x \in Ker f$. $Ker f$ is closed. 
				
				\vspace*{6em}
				
				\item \textbf{充分性$\Leftarrow$}:不妨设$Ker f \subsetneqq X$ 为$X$ 的真闭子集 (否则$f \equiv 0$ 自然连续). \\
				Consider the Quotient Space $(X / Ker f , \Vert \cdot \Vert_0) \in B^*$ (范数的定义及合理性可回顾 \textbf{Def \ref{def 2.4.1}}). \\
				下面说明$X / Ker f$ 为1维线性空间 (可将$X / Ker f$ 中的成员理解为$f$ 的等值面):\\
				Fix $\forall x_0 \not\in Kerf$, then $f(x_0) \neq 0$. Since $f(x_0) \neq 0$, then $\forall [y] \in X / Ker f$, it's clear that
				\[ [y] = \frac{f(y)}{f(x_0)} [x_0] , \,\, \forall [y] \in X / Ker f \]
				Thus $X / Ker f$ can be linearly expressed by $\{ [x_0] \} \subset X / Ker f$. 故$dim(X / Ker f) = 1$. Let
				\begin{align*}
					\widetilde{f} : X / Ker f &\longrightarrow \mathbb{K} \\
					[x] &\longmapsto f(x)
				\end{align*}
				不难证明$\widetilde{f}$ 是个线性映射 (well-defined, 与代表元无关, 保持线性运算). \\
				下面证明$\widetilde{f} \in Hom(X / Ker f , \mathbb{K})$ 有界, 即\underline{\textbf{一维$B^*$ 空间上的线性泛函均有界}}:\\
				Since $dim(X / Ker f) = 1$, then for $\forall$ fixed $x_0 \not\in Ker f$, $\exists \lambda_x \in \mathbb{K}$, $\st$
				\[ [x] = \lambda_x [x_0] = [\lambda_x x_0] , \,\, \forall [x] \in X / Ker f \]
				Suppose $\dfrac{\left| \widetilde{f}([x_0]) \right|}{\Big\Vert [x_0] \Big\Vert_0} = M \in \R_{\neq 0}$. i.e. $\left| \widetilde{f}([x_0]) \right| = M \cdot \Big\Vert [x_0] \Big\Vert_0$. Then 
				\[ \left| \widetilde{f}([x]) \right| 
				= \left| \widetilde{f}(\lambda_x [x_0]) \right| 
				= \left| \lambda_x \right| \cdot \left| \widetilde{f}([x_0]) \right| 
				= \left| \lambda_x \right| \cdot M \cdot \Big\Vert [x_0] \Big\Vert_0 
				= M \cdot \Big\Vert [x] \Big\Vert_0 , \,\, \forall [x] \in X / Ker f \]
				根据\textbf{$B^*$ 空间上有界线性算子的等价定义 (Def \ref{def 4.1.3})}, $\widetilde{f} : (X / Ker f , \Vert \cdot \Vert_0) \longrightarrow (\mathbb{K} , | \cdot |)$ 有界. \\
				从而
				\begin{align*}
					\left| f(x) \right| 
					= \left| \widetilde{f}([x]) \right| 
					= M \cdot \Big\Vert [x] \Big\Vert_0 
					= M \cdot \inf_{z \in Ker f} \Vert x - z \Vert
				\end{align*}
				Since $\overrightarrow{0} \in Ker f$, then $\underset{z \in Ker f}{\inf} \Vert x - z \Vert \leq \Vert x - \overrightarrow{0} \Vert = \Vert x \Vert$. Therefore, 
				\[ \left| f(x) \right| \leq M \cdot \Vert x \Vert , \,\, \forall x \in X \]
				从而$f \in Hom(X , \mathbb{K})$ 有界. \\
				根据\textbf{$B^*$ 空间线性算子连续与有界等价 (Prop \ref{prop 4.1.1})}, $f \in L(X , \mathbb{K}) = X^*$ 连续. 
			\end{itemize}
		\end{proof}
	\end{proposition} 


	%  ############################
	\ifx\allfiles\undefined
\end{document}
\fi