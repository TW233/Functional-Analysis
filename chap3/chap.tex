\ifx\allfiles\undefined
\input{../config/config}
\begin{document}
	% \input{../config/cover} 
	\else
	\fi
	%  ############################ 正文部分
\chapter{内积空间}

\section{内积空间}
	在先前的学习中, 我们已经得到了\textbf{度量空间}、\textbf{赋范空间}的概念, 它们的性质都很好, 但相比于我们所熟知的\textbf{欧氏空间$\R^n$}, 尤其是在\textbf{几何性质}上还存在着较大的差距. 这一章我们将给出\textbf{内积空间}的概念和性质, 并说明, 其与\textbf{赋范空间}之间只相差了一条\textbf{平行四边形公式}. 
	
	\vspace{1em}
	
	\begin{defn}\label{def 3.1.1}
		设$X$ 为定义在数域$\mathbb{K}$ 上的线性空间, 如果映射$(\cdot , \cdot)$
		\[ (\cdot , \cdot) : X \times X \longrightarrow \mathbb{K} \]
		满足如下三条性质:
		\begin{enumerate}
			\item \textbf{[正定性]}. 
			\[ (x , x) \geq 0 , \,\, \forall x \in X \hspace*{4em} \Big( (x , x) = 0 \,\, \Leftrightarrow \,\, x = 0 \Big) \]
			
			\item \textbf{[共轭对称性]}. 
			\[ (x , y) = \overline{(y , x)} , \,\, \forall x , y \in X \]
			
			\item \textbf{[关于第一变元线性性]}. 
			\[ (\alpha x_1 + \beta x_2 , y) = \alpha (x_1 , y) + \beta (x_2 , y) , \,\, \forall x_1 , x_2 , y \in X , \,\, \forall \alpha , \beta \in \mathbb{F} \]
		\end{enumerate}
		则称$(\cdot , \cdot)$ 为$X$ 上的\underline{\textcolor{blue}{\textbf{内积}}}, $(X , (\cdot , \cdot))$ 称为\underline{\textcolor{blue}{\textbf{内积空间}}}. 
		
		\vspace{3em}
		
		\begin{rmk}
			\begin{itemize}
				\item 此处讨论的\textbf{内积}应当定义在\textbf{数域$\mathbb{K}$} 上的线性空间之上, 即内积的\textbf{陪域$\mathbb{K}$} 与线性空间的域$\mathbb{K}$ 二者应当为相同的数域, 且默认取完备数域, 即$\mathbb{K} \in \{ \C , \R \}$\footnote{可参考\textbf{《Linear and Nonlinear Functional Analysis with Applications》 -- Philippe G. Ciarlet} $\S$ 4.1 Page 174, 其分别对\textbf{实内积空间}与\textbf{复内积空间}进行了定义.}. 
				
				\vspace{1em}
				
				\item $(x , 0) = (0 , x) = 0 , \,\, \forall x \in X$
				
				\newpage
				
				\item 内积关于\textbf{第二变元}具有\textbf{共轭线性性}:
				\[ (x , \alpha y_1 + \beta y_2) = \overline{\alpha} (x , y_1) + \overline{\beta} (x , y_2) , \,\, \forall x , y_1 , y_2 \in X , \,\, \forall \alpha , \beta \in \mathbb{K} \]
				
				\vspace{1em}
				
				\begin{proof}
					根据\textbf{共轭对称性}及\textbf{第一变元线性性}, $\forall x , y_1 , y_2 \in X , \,\, \forall \alpha , \beta \in \mathbb{F}$, 
					\[
						(x , \alpha y_1 + \beta y_2) 
						= \overline{(\alpha y_1 + \beta y_2 , x)} 
						= \overline{\alpha} \overline{(y_1  ,x)} + \overline{\beta} \overline{(y_2 , x)} 
						= \overline{\alpha} (x , y_1) + \overline{\beta} (x , y_2)
					\]
				\end{proof}
				
				\vspace{4em}
				
				\item 事实上, 内积可\textbf{诱导范数}, 即\textbf{内积空间为一类特殊的$B^*$ 空间}. 即
				\begin{center}
					\textbf{内积空间} $\,\, \Rightarrow \,\,$ \textbf{赋范空间} \hspace*{1em} , \hspace*{1em} \textbf{赋范空间} $\,\, \not\Rightarrow \,\,$ \textbf{内积空间}
				\end{center}
				后续我们将说明, 赋范空间只需再满足一条平行四边形公式, 即可扩充为内积空间. \\
				对于一般的内积空间$(X , (\cdot , \cdot))$, 我们总是诱导范数$\Vert \cdot \Vert$ 定义如下:
				\[ \Vert x \Vert = \sqrt{(x , x)} , \,\, \forall x \in X \]
				下面证明该定义满足\textbf{范数的三条公理 (Def \ref{def 2.1.1})}:
				
				\vspace{6em}
				
				\begin{proof}
					正定性不难验证. 对于绝对齐性, 根据\textbf{第一变元线性性}及\textbf{第二变元共轭线性性}, 
					\[ \Vert k x \Vert 
					= \sqrt{(k x , kx)} 
					= \sqrt{k(x , kx)} 
					= \sqrt{k \overline{k} (x , x)} 
					= \left| k \right| \sqrt{(x , x)} 
					= \left| k \right| \cdot \Vert x \Vert , \,\, \forall x \in X , \,\, \forall k \in \mathbb{F} \]
					而对于三角不等式, 根据我们接下来马上介绍的\textbf{Cauchy-Schwarz's Inequality (Thm \ref{thm 3.1.1})}, 
					\[ Re (x , y) \leq \Big| (x , y) \Big| \leq \Vert x \Vert \cdot \Vert y \Vert , \,\, \forall x , y \in X \]
					Thus
					\begin{align}
					\Vert x + y \Vert^2 
					= (x + y , x + y) 
					&= (x , x) + 2 Re(x , y) + (y , y) \\
					&\leq (x , x) + 2 \Vert x \Vert \cdot \Vert y \Vert + (y , y) \\
					&= \left( \Vert x \Vert + \Vert y \Vert \right)^2 , \,\, \forall x , y \in X 
					\end{align}
					i.e. 
					\[ \Vert x + y \Vert \leq \Vert x \Vert + \Vert y \Vert , \,\, \forall x , y \in X \]
					Therefore, $\Vert \cdot \Vert$ is a norm defined on $X$.
				\end{proof}
			\end{itemize}
		\end{rmk}
	\end{defn}

\newpage

\subsection{Cauchy-Schwarz's Inequality}
	下面我们将介绍大名鼎鼎的\textbf{Cauchy-Schwarz's Inequality}, 其在一般的内积空间中均成立, 并为内积空间中\textbf{“角度”}这一几何概念提供了理论支撑. 
	
	\vspace{1em}
	
	\begin{thm}\label{thm 3.1.1}
		\textbf{[Cauchy-Schwarz's Inequality]}. \\
		Suppose $(X , (\cdot , \cdot))$ be an inner product space. Let $\Vert x \Vert = \sqrt{(x , x)} , \,\, \forall x \in X$. Then 
		\[ \Big| (x , y) \Big| \leq \Vert x \Vert \cdot \Vert y \Vert , \,\, \forall x , y \in X \]
		且等号“$=$” 成立$\,\, \Leftrightarrow \,\, x , y$ 线性相关. 
		
		\vspace{6em}
		
		\begin{proof}
			下面考虑一般情况, 即数域$\mathbb{K} = \C$ 的情况 (\textbf{Def \ref{def 3.1.1}}). \\
			Fix $\forall x,  y \in X$. Let $f : \C \longrightarrow \R_{\geq 0}$, 
			\begin{align} 
			f(t) 
			= \Vert x + ty \Vert^2 
			&= (x , x) + 2 Re(x , ty) + \left| t \right|^2 (y , y) \\
			&= (x , x) + 2 Re \Big( \overline{t} (x , y) \Big) + \left| t \right|^2 (y , y)
			\end{align}
			Then $f(t) \geq 0 , \,\, \forall t \in \C$. 下面我们只考虑$\overline{t} (x , y) \in \R$ 的情形, 即
			\[ t = s \frac{(x , y)}{\left| (x , y) \right|} , \,\, \forall s \in \R \]
			Thus $2Re \Big( \overline{t} (x , y) \Big) = 2 Re \Big( s \cdot \dfrac{\overline{(x , y)}}{\left| (x , y) \right|} \cdot (x , y) \Big) = 2s \left| (x , y) \right|$. Then we have
			\[
				f(t) = g(s) = (x , x) + 2s \left| (x , y) \right| + s^2 (y , y) , \,\, \forall s \in \R
			\]
			Since $f(t) \geq 0 , \,\, \forall t \in \C$, then $g(s) \geq 0 , \,\, \forall s \in \R$. Thus
			\[ \Delta_g = 4\Big| (x , y) \Big|^2 - 4 (x , x) (y , y) \leq 0 \]
			i.e. 
			\[ \Big| (x , y) \Big| \leq \Vert x \Vert \cdot \Vert y \Vert , \,\, \forall x , y \in X \]
		\end{proof}
	\end{thm}

\newpage

\section{内积与范数的关系}
	这一节我们将揭示\textbf{内积}与\textbf{范数}的关系, 即内积空间为特殊的$B^*$ 空间, 但二者事实上只相差一个\textbf{平行四边形公式}. 
	
\subsection{内积的连续性}
	下面我们来说明, 内积$(\cdot , \cdot)$ 关于其所诱导的范数连续. 
	
	\vspace{1em}
	
	\begin{proposition}\label{prop 3.2.1}
		\textbf{[内积的连续性]}. 
		\begin{center}
			\textbf{对于内积空间$(X , (\cdot , \cdot))$, 其内积$(\cdot , \cdot)$ 在$X \times X$ 上关于其诱导的范数$\Vert \cdot \Vert$ 连续}.
		\end{center}
		
		\vspace{2em}
		
		\begin{rmk}
			该命题严谨叙述应该为$(\cdot , \cdot)$ 在其诱导的范数$\Vert \cdot \Vert$ 所诱导的度量$\rho(\cdot , \cdot)$ 下连续, 即连续的概念应当建立在拓扑上, 特别地可为度量所诱导的拓扑. 
		\end{rmk}
		
		\vspace{4em}
		
		\begin{proof}
			Suppose $x_n \overset{\Vert \cdot \Vert}{\to} x \in X$, $y_n \overset{\Vert \cdot \Vert}{\to} y \in X$. Then both $\{ x_n \}_{n = 1}^{\infty}$ and $\{ y_n \}_{n = 1}^{\infty}$ are bounded. \\
			i.e. $\exists M > 0$, $\st$
			\[ \Vert x_n \Vert \leq M , \,\, \Vert y_n \Vert \leq M , \,\, \forall n \in \N \]
			Therefore, 
			\begin{align}
				\Big| (x_n , y_n) - (x , y) \Big| 
				&= \Big| (x_n , y_n) - (x , y_n) + (x , y_n) - (x , y) \Big| \\
				&\leq \Big| (x_n - x , y_n) \Big| + \Big| (x , y_n - y) \Big|
			\end{align}
			By \textbf{Cauchy-Schwarz's Inequality (Thm \ref{thm 3.1.1})}, 
			\begin{align}
				\Big| (x_n , y_n) - (x , y) \Big| 
				&\leq \Big| (x_n - x , y_n) \Big| + \Big| (x , y_n - y) \Big| \\
				&\leq \Vert x_n - x \Vert \cdot \Vert y_n \Vert + \Vert x \Vert \cdot \Vert y_n - y \Vert \\
				&\leq M \cdot \Vert x_n - x \Vert + \Vert x \Vert \cdot \Vert y_n - y \Vert \to 0
			\end{align}
			Therefore, $(x_n , y_n) \to (x , y)$ in $\mathbb{K}$. $(\cdot , \cdot) : X \times X \longrightarrow \mathbb{K}$ is continuous. 
		\end{proof}
	\end{proposition}

\newpage

\subsection{极化恒等式与平行四边形公式}
	本小节将说明赋范空间配备上\textbf{平行四边形公式}后即可成为内积空间, 并给出内积的定义式 -- \textbf{极化恒等式}. 首先来给出内积空间的\textbf{极化恒等式}. 
	
	\vspace{1em}
	
	\begin{proposition}\label{prop 3.2.2}
		\textbf{[极化恒等式]}. \\
		设$(X , (\cdot , \cdot))$ 为定义在数域$\mathbb{K}$ 上的内积空间, 则 
		\begin{enumerate}
			\item[(\rmnum{1})] If $\mathbb{K} = \R$, then 
			\[ (x , y) = \dfrac{1}{4} \Big( \Vert x + y \Vert^2 - \Vert x - y \Vert^2 \Big) , \,\, \forall x , y \in X \]
			
			\item[(\rmnum{2})] If $\mathbb{K} = \C$, then
			\[ (x , y) = \dfrac{1}{4} \Big( \Vert x + y \Vert^2 - \Vert x - y \Vert^2 \Big) + \dfrac{i}{4} \Big( \Vert x + iy \Vert^2 - \Vert x - iy \Vert^2 \Big) , \,\, \forall x , y \in X \]
		\end{enumerate}
		
		\vspace{6em}
		
		\begin{proof}
			当$\mathbb{K} = \C$ 时, 
			\begin{align}
				\frac{1}{4} \Big( \Vert x + y \Vert^2 - \Vert x - y \Vert^2 \Big) 
				&= Re(x , y) \\
				\frac{i}{4} \Big( \Vert x + iy \Vert^2 - \Vert x - iy \Vert^2 \Big) 
				&= i Re(x , iy) , \,\, \forall x , y \in X
			\end{align}
			而
			\[ Re(x , iy) = Re \Big( \bar{i} (x , y) \Big) = Im(x , y) , \,\, \forall x , y \in X \]
			Therefore, 
			\begin{align}
				&\frac{1}{4} \Big( \Vert x + y \Vert^2 - \Vert x - y \Vert^2 \Big) + \dfrac{i}{4} \Big( \Vert x + iy \Vert^2 - \Vert x - iy \Vert^2 \Big) \\
				&= Re(x , y) + i Im(x , y) \\
				&= (x , y) , \,\, \forall x , y \in X
			\end{align}
		\end{proof}
	\end{proposition}

	\vspace{4em}
	
	下面我们将说明, 对于$B^*$ 空间$(X , \Vert \cdot \Vert)$, 若其范数满足\textbf{平行四边形公式}, 则可引入内积$(\cdot , \cdot)$, $\st$
	\[ \sqrt{(x , x)} = \Vert x \Vert , \,\, \forall x \in X \]
	
	\newpage
	
	\begin{thm}\label{thm 3.2.1}
		\textbf{[范数诱导内积]}. \\
		设$(X , \Vert \cdot \Vert) \in B^*$, 若其范数满足平行四边形公式, 即
		\[ \Vert x + y \Vert^2 + \Vert x - y \Vert^2 = 2 \Big( \Vert x \Vert^2 + \Vert y \Vert^2 \Big) , \,\, \forall x , y \in X \]
		则可定义由此范数诱导的内积$(\cdot , \cdot)$, $\st \sqrt{(x , x)} = \Vert x \Vert , \,\, \forall x \in X$, 使之成为内积空间. 
		
		\vspace{4em}
		
		\begin{proof}
			内积按照\textbf{极化恒等式 (Prop \ref{prop 3.2.2})}定义. \\
			下面我们只证明$\mathbb{K} = \R$ 的情形, $\mathbb{K} = \C$ 可类似证明. 即定义
			\[ (x , y) \coloneqq \frac{1}{4} \Big( \Vert x + y \Vert^2 - \Vert x - y \Vert^2 \Big) , \forall x , y \in X \]
			依次验证\textbf{内积的三条公理 (Def \ref{def 3.1.1})}, \textbf{正定性}及\textbf{共轭对称性}显然成立. \\ 
			下面分两方面来证明\textbf{关于第一变元线性性}:(对加法$\&$ 数乘封闭) 
			
			\vspace{1em}
			
			\begin{itemize}
				\item \textbf{对加法封闭}:Since
				\begin{align}
					(x , z) + (y , z) 
					&= \frac{1}{4} \Big( \Vert x + z \Vert^2 - \Vert x - z \Vert^2 + \Vert y + z \Vert^2 - \Vert y - z \Vert^2 \Big) \\
					&= \frac{1}{8} \Big[ 2 \Big( \Vert x + z \Vert^2 + \Vert y + z \Vert^2 \Big) - 2 \Big( \Vert x - z \Vert^2 + \Vert y - z \Vert^2 \Big) \Big]
				\end{align}
				By \textbf{平行四边形公式}, 
				\begin{align}
					(x , z) + (y , z) 
					&= \frac{1}{8} \Big[ 2 \Big( \Vert x + z \Vert^2 + \Vert y + z \Vert^2 \Big) - 2 \Big( \Vert x - z \Vert^2 + \Vert y - z \Vert^2 \Big) \Big] \\
					&= \frac{1}{8} \Big( \Vert x + y + 2z \Vert^2 + \Vert x - y \Vert^2 - \Vert x + y - 2z \Vert^2 - \Vert x - y \Vert^2 \Big) \\ 
					&= \frac{1}{8} \Big( \Vert x + y + 2z \Vert^2 - \Vert x + y - 2z \Vert^2 \Big) \\
					&= \frac{1}{2} (x + y , 2z) 
					= 2 \left( \frac{x + y}{2} , z \right) , \,\, \forall x , y , z \in X
				\end{align}
				Let $y = 0$, we get
				\[ (x , z) = 2 \left( \frac{x}{2} , z \right) , \,\, \forall x , z \in X \]
				Thus replace $x$ by $x + y$, 
				\[ (x , z) + (y , z) = 2 \left( \frac{x + y}{2} , z \right) = (x + y , z) , \,\, \forall x , y , z \in X \]
				i.e. 
				\[ (x , z) + (y , z) = (x + y , z) , \,\, \forall x , y , z \in X \]
				
				\newpage
				
				\item \textbf{对数乘封闭}:Fix $\forall x , y \in X$. Let
				\begin{align}
					f : \R &\longrightarrow \R \\
					t &\longmapsto f(t) = (tx , y)
				\end{align}
				只需证:$f(t)$ 为线性函数. \\
				By the previous result, 
				\[ f(t_1 + t_2) = f(t_1) + f(t_2) , \,\, \forall t_1 , t_2 \in \R \]
				Since $\Vert \cdot \Vert$ is continuous (\textbf{范数的连续性 (Def \ref{def 2.1.1})}), then $f$ is continuous. \\
				Thus $f$ 为线性函数, i.e.
				\[ f(t) = tf(1) , \,\, \forall t \in \R \]
				Therefore, 
				\[ (tx , y) = t(x , y) , \,\, \forall t \in \R \]
			\end{itemize}
			
			综上, 
			\[ (\alpha x + \beta y , z) = \alpha (x , z) + \beta (y , z) , \,\, \forall x , y , z \in X , \,\, \forall \alpha , \beta \in \R \]
		\end{proof}
	\end{thm}
	
















	%  ############################
	\ifx\allfiles\undefined
\end{document}
\fi