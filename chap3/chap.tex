\ifx\allfiles\undefined
\input{../config/config}
\begin{document}
	% \input{../config/cover} 
	\else
	\fi
	%  ############################ 正文部分
\chapter{内积空间}

\section{内积空间}
	在先前的学习中, 我们已经得到了\textbf{度量空间}、\textbf{赋范空间}的概念, 它们的性质都很好, 但相比于我们所熟知的\textbf{欧氏空间$\R^n$}, 尤其是在\textbf{几何性质}上还存在着较大的差距. 这一章我们将给出\textbf{内积空间}的概念和性质, 并说明, 其与\textbf{赋范空间}之间只相差了一条\textbf{平行四边形公式}. 
	
	\vspace{1em}
	
	\begin{defn}\label{def 3.1.1}
		设$X$ 为定义在域$\mathbb{K}$ 上的线性空间, 如果映射$(\cdot , \cdot)$
		\[ (\cdot , \cdot) : X \times X \longrightarrow \mathbb{K} \]
		满足如下三条性质:
		\begin{enumerate}
			\item \textbf{[正定性]}. 
			\[ (x , x) \geq 0 , \,\, \forall x \in X \hspace*{4em} \Big( (x , x) = 0 \,\, \Leftrightarrow \,\, x = 0 \Big) \]
			
			\item \textbf{[共轭对称性]}. 
			\[ (x , y) = \overline{(y , x)} , \,\, \forall x , y \in X \]
			
			\item \textbf{[关于第一变元线性性]}. 
			\[ (\alpha x_1 + \beta x_2 , y) = \alpha (x_1 , y) + \beta (x_2 , y) , \,\, \forall x_1 , x_2 , y \in X , \,\, \forall \alpha , \beta \in \mathbb{K} \]
		\end{enumerate}
		则称$(\cdot , \cdot)$ 为$X$ 上的\underline{\textcolor{blue}{\textbf{内积}}}, $(X , (\cdot , \cdot))$ 称为\underline{\textcolor{blue}{\textbf{内积空间}}}. 
		
		\vspace{4em}
		
		\begin{rmk}
			\begin{itemize}
				\item $(x , 0) = (0 , x) = 0 , \,\, \forall x \in X$
				
				\vspace{3em}
				
				\item 内积关于\textbf{第二变元}具有\textbf{共轭线性性}:
				\[ (x , \alpha y_1 + \beta y_2) = \overline{\alpha} (x , y_1) + \overline{\beta} (x , y_2) , \,\, \forall x , y_1 , y_2 \in X , \,\, \forall \alpha , \beta \in \mathbb{K} \]
				
				\newpage
				
				\begin{proof}
					根据\textbf{共轭对称性}及\textbf{第一变元线性性}, $\forall x , y_1 , y_2 \in X , \,\, \forall \alpha , \beta \in \mathbb{F}$, 
					\[
						(x , \alpha y_1 + \beta y_2) 
						= \overline{(\alpha y_1 + \beta y_2 , x)} 
						= \overline{\alpha} \overline{(y_1  ,x)} + \overline{\beta} \overline{(y_2 , x)} 
						= \overline{\alpha} (x , y_1) + \overline{\beta} (x , y_2)
					\]
				\end{proof}
				
				\vspace{4em}
				
				\item 事实上, 内积可\textbf{诱导范数}, 即\textbf{内积空间为一类特殊的$B^*$ 空间}. 即
				\begin{center}
					\textbf{内积空间} $\,\, \Rightarrow \,\,$ \textbf{赋范空间} \hspace*{1em} , \hspace*{1em} \textbf{赋范空间} $\,\, \not\Rightarrow \,\,$ \textbf{内积空间}
				\end{center}
				后续我们将说明, 赋范空间只需再满足一条平行四边形公式, 即可扩充为内积空间. \\
				对于一般的内积空间$(X , (\cdot , \cdot))$, 我们总是诱导范数$\Vert \cdot \Vert$ 定义如下:
				\[ \Vert x \Vert = \sqrt{(x , x)} , \,\, \forall x \in X \]
				下面证明该定义满足\textbf{范数的三条公理 (Def \ref{def 2.1.1})}:
				
				\vspace{6em}
				
				\begin{proof}
					正定性不难验证. 对于绝对齐性, 根据\textbf{第一变元线性性}及\textbf{第二变元共轭线性性}, 
					\[ \Vert k x \Vert 
					= \sqrt{(k x , kx)} 
					= \sqrt{k(x , kx)} 
					= \sqrt{k \overline{k} (x , x)} 
					= \left| k \right| \sqrt{(x , x)} 
					= \left| k \right| \cdot \Vert x \Vert , \,\, \forall x \in X , \,\, \forall k \in \mathbb{F} \]
					而对于三角不等式, 根据我们接下来马上介绍的\textbf{Cauchy-Schwarz's Inequality (Thm \ref{thm 3.1.1})}, 
					\[ Re (x , y) \leq \Big| (x , y) \Big| \leq \Vert x \Vert \cdot \Vert y \Vert , \,\, \forall x , y \in X \]
					Thus
					\begin{align}
					\Vert x + y \Vert^2 
					= (x + y , x + y) 
					&= (x , x) + 2 Re(x , y) + (y , y) \\
					&\leq (x , x) + 2 \Vert x \Vert \cdot \Vert y \Vert + (y , y) \\
					&= \left( \Vert x \Vert + \Vert y \Vert \right)^2 , \,\, \forall x , y \in X 
					\end{align}
					i.e. 
					\[ \Vert x + y \Vert \leq \Vert x \Vert + \Vert y \Vert , \,\, \forall x , y \in X \]
					Therefore, $\Vert \cdot \Vert$ is a norm defined on $X$.
				\end{proof}
			\end{itemize}
		\end{rmk}
	\end{defn}

\newpage

\subsection{Cauchy-Schwarz's Inequality}
	下面我们将介绍大名鼎鼎的\textbf{Cauchy-Schwarz's Inequality}, 其在一般的内积空间中均成立, 并为内积空间中\textbf{“角度”}这一几何概念提供了理论支撑. 
	
	\vspace{1em}
	
	\begin{thm}\label{thm 3.1.1}
		\textbf{[Cauchy-Schwarz's Inequality]}. \\
		Suppose $(X , (\cdot , \cdot))$ be an inner product space. Let $\Vert x \Vert = \sqrt{(x , x)} , \,\, \forall x \in X$. Then 
		\[ \Big| (x , y) \Big| \leq \Vert x \Vert \cdot \Vert y \Vert , \,\, \forall x , y \in X \]
		且等号“$=$” 成立$\,\, \Leftrightarrow \,\, x , y$ 线性相关. 
		
		\vspace{6em}
		
		\begin{proof}
			下面考虑一般情况, 即域$\mathbb{K} = \C$ 的情况 (\textbf{Def \ref{def 3.1.1}}). \\
			Fix $\forall x,  y \in X$. Let $f : \C \longrightarrow \R_{\geq 0}$, 
			\begin{align} 
			f(t) 
			= \Vert x + ty \Vert^2 
			&= (x , x) + 2 Re(x , ty) + \left| t \right|^2 (y , y) \\
			&= (x , x) + 2 Re \Big( \overline{t} (x , y) \Big) + \left| t \right|^2 (y , y)
			\end{align}
			Then $f(t) \geq 0 , \,\, \forall t \in \C$. 下面我们只考虑$\overline{t} (x , y) \in \R$ 的情形, 即
			\[ t = s \frac{(x , y)}{\left| (x , y) \right|} , \,\, \forall s \in \R \]
			Thus $2Re \Big( \overline{t} (x , y) \Big) = 2 Re \Big( s \cdot \dfrac{\overline{(x , y)}}{\left| (x , y) \right|} \cdot (x , y) \Big) = 2s \left| (x , y) \right|$. Then we have
			\[
				f(t) = g(s) = (x , x) + 2s \left| (x , y) \right| + s^2 (y , y) , \,\, \forall s \in \R
			\]
			Since $f(t) \geq 0 , \,\, \forall t \in \C$, then $g(s) \geq 0 , \,\, \forall s \in \R$. Thus
			\[ \Delta_g = 4\Big| (x , y) \Big|^2 - 4 (x , x) (y , y) \leq 0 \]
			i.e. 
			\[ \Big| (x , y) \Big| \leq \Vert x \Vert \cdot \Vert y \Vert , \,\, \forall x , y \in X \]
		\end{proof}
	\end{thm}
















	%  ############################
	\ifx\allfiles\undefined
\end{document}
\fi