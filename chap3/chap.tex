\ifx\allfiles\undefined
\documentclass[12pt, a4paper,oneside, UTF8]{ctexbook}
\usepackage[dvipsnames]{xcolor}
\usepackage{mathtools}   % 数学公式
\usepackage{amsthm}    % 定理环境
\usepackage{amssymb}   % 更多公式符号
\usepackage{graphicx}  % 插图
%\usepackage{mathrsfs}  % 数学字体
%\usepackage{newtxtext,newtxmath}
%\usepackage{arev}
\usepackage{kmath,kerkis}
\usepackage{newtxtext}
\usepackage{bbm}
\usepackage{enumitem}  % 列表
\usepackage{geometry}  % 页面调整
%\usepackage{unicode-math}
\usepackage[colorlinks,linkcolor=black]{hyperref}

\usepackage{wrapfig}


\usepackage{ulem}	   % 用于更多的下划线格式,
					   % \uline{}下划线,\uuline{}双下划线,\uwave{}下划波浪线,\sout{}中间删除线,\xout{}斜删除线
					   % \dashuline{}下划虚线,\dotuline{}文字底部加点


\graphicspath{ {flg/},{../flg/}, {config/}, {../config/} }  % 配置图形文件检索目录
\linespread{1.5} % 行高

% 页码设置
\geometry{top=25.4mm,bottom=25.4mm,left=20mm,right=20mm,headheight=2.17cm,headsep=4mm,footskip=12mm}

% 设置列表环境的上下间距
\setenumerate[1]{itemsep=5pt,partopsep=0pt,parsep=\parskip,topsep=5pt}
\setitemize[1]{itemsep=5pt,partopsep=0pt,parsep=\parskip,topsep=5pt}
\setdescription{itemsep=5pt,partopsep=0pt,parsep=\parskip,topsep=5pt}

% 定理环境
% ########## 定理环境 start ####################################
\theoremstyle{definition}
\newtheorem{defn}{\indent 定义}[section]

\newtheorem{lemma}{\indent 引理}[section]    % 引理 定理 推论 准则 共用一个编号计数
\newtheorem{thm}[lemma]{\indent 定理}
\newtheorem{corollary}[lemma]{\indent 推论}
\newtheorem{criterion}[lemma]{\indent 准则}

\newtheorem{proposition}{\indent 命题}[section]
\newtheorem{example}{\indent \color{SeaGreen}{例}}[section] % 绿色文字的 例 ,不需要就去除\color{SeaGreen}{}
\newtheorem*{rmk}{\indent \color{red}{注}}

% 两种方式定义中文的 证明 和 解 的环境:
% 缺点:\qedhere 命令将会失效【技术有限,暂时无法解决】
\renewenvironment{proof}{\par\textbf{证明.}\;}{\qed\par}
\newenvironment{solution}{\par{\textbf{解.}}\;}{\qed\par}

% 缺点:\bf 是过时命令,可以用 textb f等替代,但编译会有关于字体的警告,不过不影响使用【技术有限,暂时无法解决】
%\renewcommand{\proofname}{\indent\bf 证明}
%\newenvironment{solution}{\begin{proof}[\indent\bf 解]}{\end{proof}}
% ######### 定理环境 end  #####################################

% ↓↓↓↓↓↓↓↓↓↓↓↓↓↓↓↓↓ 以下是自定义的命令  ↓↓↓↓↓↓↓↓↓↓↓↓↓↓↓↓

% 用于调整表格的高度  使用 \hline\xrowht{25pt}
\newcommand{\xrowht}[2][0]{\addstackgap[.5\dimexpr#2\relax]{\vphantom{#1}}}

% 表格环境内长内容换行
\newcommand{\tabincell}[2]{\begin{tabular}{@{}#1@{}}#2\end{tabular}}

% 使用\linespread{1.5} 之后 cases 环境的行高也会改变,重新定义一个 ca 环境可以自动控制 cases 环境行高
\newenvironment{ca}[1][1]{\linespread{#1} \selectfont \begin{cases}}{\end{cases}}
% 和上面一样
\newenvironment{vx}[1][1]{\linespread{#1} \selectfont \begin{vmatrix}}{\end{vmatrix}}

\def\d{\textup{d}} % 直立体 d 用于微分符号 dx
\def\R{\mathbb{R}} % 实数域
\def\N{\mathbb{N}} % 自然数域
\def\C{\mathbb{C}} % 复数域
\def\Z{\mathbb{Z}} % 整数环
\def\Q{\mathbb{Q}} % 有理数域
\newcommand{\bs}[1]{\boldsymbol{#1}}    % 加粗,常用于向量
\newcommand{\ora}[1]{\overrightarrow{#1}} % 向量

% 数学 平行 符号
\newcommand{\pll}{\kern 0.56em/\kern -0.8em /\kern 0.56em}

% 用于空行\myspace{1} 表示空一行 填 2 表示空两行  
\newcommand{\myspace}[1]{\par\vspace{#1\baselineskip}}

%s.t. 用\st就能打出s.t.
\DeclareMathOperator{\st}{s.t.}

%罗马数字 \rmnum{}是小写罗马数字, \Rmnum{}是大写罗马数字
\makeatletter
\newcommand{\rmnum}[1]{\romannumeral #1}
\newcommand{\Rmnum}[1]{\expandafter@slowromancap\romannumeral #1@}
\makeatother
\begin{document}
	% \title{{\Huge{\textbf{$Functional \,\, Analysis$}}}\footnote{参考书籍:\\
			\hspace*{4em} \textbf{《Linear and Nonlinear Functional Analysis with Applications》 -- Philippe G. Ciarlet} \\
			\hspace*{4em} \textbf{《Real  Analysis -- Modern Techniques and Their Applications》 -- Gerald  B.  Folland} \\
			\hspace*{4em} \textbf{《Functional Analysis -- Introduction to Further Topics in Analysis》 -- Elias M. Stein} \\
			\hspace*{4em} \textbf{《泛函分析讲义》 -- 张恭庆、林源渠} 
			}}
\author{$-TW-$}
\date{\today}
\maketitle                   % 在单独的标题页上生成一个标题

\thispagestyle{empty}        % 前言页面不使用页码
\begin{center}
	\Huge\textbf{序}
\end{center}


\vspace*{3em}
\begin{center}
	\large{\textbf{天道几何,万品流形先自守;}}\\
	
	\large{\textbf{变分无限,孤心测度有同伦。}}
\end{center}

\vspace*{3em}
\begin{flushright}
	\begin{tabular}{c}
		\today \\ \small{\textbf{长夜伴浪破晓梦,梦晓破浪伴夜长}}
	\end{tabular}
\end{flushright}


\newpage                      % 新的一页
\pagestyle{plain}             % 设置页眉和页脚的排版方式(plain:页眉是空的,页脚只包含一个居中的页码)
\setcounter{page}{1}          % 重新定义页码从第一页开始
\pagenumbering{Roman}         % 使用大写的罗马数字作为页码
\tableofcontents              % 生成目录

\newpage                      % 以下是正文
\pagestyle{plain}
\setcounter{page}{1}          % 使用阿拉伯数字作为页码
\pagenumbering{arabic}
\setcounter{chapter}{0}    % 设置 -1 可作为第零章绪论从第零章开始 
	\else
	\fi
	%  ############################ 正文部分
\chapter{内积空间}

\section{内积空间}
	在先前的学习中, 我们已经得到了\textbf{度量空间}、\textbf{赋范空间}的概念, 它们的性质都很好, 但相比于我们所熟知的\textbf{欧氏空间$\R^n$}, 尤其是在\textbf{几何性质}上还存在着较大的差距. 这一章我们将给出\textbf{内积空间}的概念和性质, 并说明, 其与\textbf{赋范空间}之间只相差了一条\textbf{平行四边形公式}. 
	
	\vspace{1em}
	
	\begin{defn}\label{def 3.1.1}
		设$X$ 为定义在数域$\mathbb{K}$ 上的线性空间, 如果映射$(\cdot , \cdot)$
		\[ (\cdot , \cdot) : X \times X \longrightarrow \mathbb{K} \]
		满足如下三条性质:
		\begin{enumerate}
			\item \textbf{[正定性]}. 
			\[ (x , x) \geq 0 , \,\, \forall x \in X \hspace*{4em} \Big( (x , x) = 0 \,\, \Leftrightarrow \,\, x = 0 \Big) \]
			
			\item \textbf{[共轭对称性]}. 
			\[ (x , y) = \overline{(y , x)} , \,\, \forall x , y \in X \]
			
			\item \textbf{[关于第一变元线性性]}. 
			\[ (\alpha x_1 + \beta x_2 , y) = \alpha (x_1 , y) + \beta (x_2 , y) , \,\, \forall x_1 , x_2 , y \in X , \,\, \forall \alpha , \beta \in \mathbb{F} \]
		\end{enumerate}
		则称$(\cdot , \cdot)$ 为$X$ 上的\underline{\textcolor{blue}{\textbf{内积}}}, $(X , (\cdot , \cdot))$ 称为\underline{\textcolor{blue}{\textbf{内积空间}}}. 
		
		\vspace{3em}
		
		\begin{rmk}
			\begin{itemize}
				\item 此处讨论的\textbf{内积}应当定义在\textbf{数域$\mathbb{K}$} 上的线性空间之上, 即内积的\textbf{陪域$\mathbb{K}$} 与线性空间的域$\mathbb{K}$ 二者应当为相同的数域, 且默认取完备数域, 即$\mathbb{K} \in \{ \C , \R \}$\footnote{可参考\textbf{《Linear and Nonlinear Functional Analysis with Applications》 -- Philippe G. Ciarlet} $\S$ 4.1 Page 174, 其分别对\textbf{实内积空间}与\textbf{复内积空间}进行了定义.}. 
				
				\vspace{1em}
				
				\item $(x , 0) = (0 , x) = 0 , \,\, \forall x \in X$
				
				\newpage
				
				\item 内积关于\textbf{第二变元}具有\textbf{共轭线性性}:
				\[ (x , \alpha y_1 + \beta y_2) = \overline{\alpha} (x , y_1) + \overline{\beta} (x , y_2) , \,\, \forall x , y_1 , y_2 \in X , \,\, \forall \alpha , \beta \in \mathbb{K} \]
				
				\vspace{1em}
				
				\begin{proof}
					根据\textbf{共轭对称性}及\textbf{第一变元线性性}, $\forall x , y_1 , y_2 \in X , \,\, \forall \alpha , \beta \in \mathbb{F}$, 
					\[
						(x , \alpha y_1 + \beta y_2) 
						= \overline{(\alpha y_1 + \beta y_2 , x)} 
						= \overline{\alpha} \overline{(y_1  ,x)} + \overline{\beta} \overline{(y_2 , x)} 
						= \overline{\alpha} (x , y_1) + \overline{\beta} (x , y_2)
					\]
				\end{proof}
				
				\vspace{4em}
				
				\item 事实上, 内积可\textbf{诱导范数}, 即\textbf{内积空间为一类特殊的$B^*$ 空间}. 即
				\begin{center}
					\textbf{内积空间} $\,\, \Rightarrow \,\,$ \textbf{赋范空间} \hspace*{1em} , \hspace*{1em} \textbf{赋范空间} $\,\, \not\Rightarrow \,\,$ \textbf{内积空间}
				\end{center}
				后续我们将说明, 赋范空间只需再满足一条平行四边形公式, 即可扩充为内积空间. \\
				对于一般的内积空间$(X , (\cdot , \cdot))$, 我们总是诱导范数$\Vert \cdot \Vert$ 定义如下:
				\[ \Vert x \Vert = \sqrt{(x , x)} , \,\, \forall x \in X \]
				下面证明该定义满足\textbf{范数的三条公理 (Def \ref{def 2.1.1})}:
				
				\vspace{6em}
				
				\begin{proof}
					正定性不难验证. 对于绝对齐性, 根据\textbf{第一变元线性性}及\textbf{第二变元共轭线性性}, 
					\[ \Vert k x \Vert 
					= \sqrt{(k x , kx)} 
					= \sqrt{k(x , kx)} 
					= \sqrt{k \overline{k} (x , x)} 
					= \left| k \right| \sqrt{(x , x)} 
					= \left| k \right| \cdot \Vert x \Vert , \,\, \forall x \in X , \,\, \forall k \in \mathbb{F} \]
					而对于三角不等式, 根据我们接下来马上介绍的\textbf{Cauchy-Schwarz's Inequality (Thm \ref{thm 3.1.1})}, 
					\[ Re (x , y) \leq \Big| (x , y) \Big| \leq \Vert x \Vert \cdot \Vert y \Vert , \,\, \forall x , y \in X \]
					Thus
					\begin{align}
					\Vert x + y \Vert^2 
					= (x + y , x + y) 
					&= (x , x) + 2 Re(x , y) + (y , y) \\
					&\leq (x , x) + 2 \Vert x \Vert \cdot \Vert y \Vert + (y , y) \\
					&= \left( \Vert x \Vert + \Vert y \Vert \right)^2 , \,\, \forall x , y \in X 
					\end{align}
					i.e. 
					\[ \Vert x + y \Vert \leq \Vert x \Vert + \Vert y \Vert , \,\, \forall x , y \in X \]
					Therefore, $\Vert \cdot \Vert$ is a norm defined on $X$.
				\end{proof}
			\end{itemize}
		\end{rmk}
	\end{defn}

\newpage

\subsection{Cauchy-Schwarz's Inequality}
	下面我们将介绍大名鼎鼎的\textbf{Cauchy-Schwarz's Inequality}, 其在一般的内积空间中均成立, 并为内积空间中\textbf{“角度”}这一几何概念提供了理论支撑. 
	
	\vspace{1em}
	
	\begin{thm}\label{thm 3.1.1}
		\textbf{[Cauchy-Schwarz's Inequality]}. \\
		Suppose $(X , (\cdot , \cdot))$ be an inner product space. Let $\Vert x \Vert = \sqrt{(x , x)} , \,\, \forall x \in X$. Then 
		\[ \Big| (x , y) \Big| \leq \Vert x \Vert \cdot \Vert y \Vert , \,\, \forall x , y \in X \]
		且等号“$=$” 成立$\,\, \Leftrightarrow \,\, x , y$ 线性相关. 
		
		\vspace{6em}
		
		\begin{proof}
			下面考虑一般情况, 即数域$\mathbb{K} = \C$ 的情况 (\textbf{Def \ref{def 3.1.1}}). \\
			Fix $\forall x,  y \in X$. Let $f : \C \longrightarrow \R_{\geq 0}$, 
			\begin{align} 
			f(t) 
			= \Vert x + ty \Vert^2 
			&= (x , x) + 2 Re(x , ty) + \left| t \right|^2 (y , y) \\
			&= (x , x) + 2 Re \Big( \overline{t} (x , y) \Big) + \left| t \right|^2 (y , y)
			\end{align}
			Then $f(t) \geq 0 , \,\, \forall t \in \C$. 下面我们只考虑$\overline{t} (x , y) \in \R$ 的情形, 即
			\[ t = s \frac{(x , y)}{\left| (x , y) \right|} , \,\, \forall s \in \R \]
			Thus $2Re \Big( \overline{t} (x , y) \Big) = 2 Re \Big( s \cdot \dfrac{\overline{(x , y)}}{\left| (x , y) \right|} \cdot (x , y) \Big) = 2s \left| (x , y) \right|$. Then we have
			\[
				f(t) = g(s) = (x , x) + 2s \left| (x , y) \right| + s^2 (y , y) , \,\, \forall s \in \R
			\]
			Since $f(t) \geq 0 , \,\, \forall t \in \C$, then $g(s) \geq 0 , \,\, \forall s \in \R$. Thus
			\[ \Delta_g = 4\Big| (x , y) \Big|^2 - 4 (x , x) (y , y) \leq 0 \]
			i.e. 
			\[ \Big| (x , y) \Big| \leq \Vert x \Vert \cdot \Vert y \Vert , \,\, \forall x , y \in X \]
		\end{proof}
	\end{thm}

\newpage

\section{内积与范数的关系}
	这一节我们将揭示\textbf{内积}与\textbf{范数}的关系, 即内积空间为特殊的$B^*$ 空间, 但二者事实上只相差一个\textbf{平行四边形公式}. 
	
\subsection{内积的连续性}
	下面我们来说明, 内积$(\cdot , \cdot)$ 关于其所诱导的范数连续. 
	
	\vspace{1em}
	
	\begin{proposition}\label{prop 3.2.1}
		\textbf{[内积的连续性]}. 
		\begin{center}
			\textbf{对于内积空间$(X , (\cdot , \cdot))$, 其内积$(\cdot , \cdot)$ 在$X \times X$ 上关于其诱导的范数$\Vert \cdot \Vert$ 连续}.
		\end{center}
		
		\vspace{2em}
		
		\begin{rmk}
			该命题严谨叙述应该为$(\cdot , \cdot)$ 在其诱导的范数$\Vert \cdot \Vert$ 所诱导的度量$\rho(\cdot , \cdot)$ 下连续, 即连续的概念应当建立在拓扑上, 特别地可为度量所诱导的拓扑. 
		\end{rmk}
		
		\vspace{4em}
		
		\begin{proof}
			Suppose $x_n \overset{\Vert \cdot \Vert}{\to} x \in X$, $y_n \overset{\Vert \cdot \Vert}{\to} y \in X$. Then both $\{ x_n \}_{n = 1}^{\infty}$ and $\{ y_n \}_{n = 1}^{\infty}$ are bounded. \\
			i.e. $\exists M > 0$, $\st$
			\[ \Vert x_n \Vert \leq M , \,\, \Vert y_n \Vert \leq M , \,\, \forall n \in \N \]
			Therefore, 
			\begin{align}
				\Big| (x_n , y_n) - (x , y) \Big| 
				&= \Big| (x_n , y_n) - (x , y_n) + (x , y_n) - (x , y) \Big| \\
				&\leq \Big| (x_n - x , y_n) \Big| + \Big| (x , y_n - y) \Big|
			\end{align}
			By \textbf{Cauchy-Schwarz's Inequality (Thm \ref{thm 3.1.1})}, 
			\begin{align}
				\Big| (x_n , y_n) - (x , y) \Big| 
				&\leq \Big| (x_n - x , y_n) \Big| + \Big| (x , y_n - y) \Big| \\
				&\leq \Vert x_n - x \Vert \cdot \Vert y_n \Vert + \Vert x \Vert \cdot \Vert y_n - y \Vert \\
				&\leq M \cdot \Vert x_n - x \Vert + \Vert x \Vert \cdot \Vert y_n - y \Vert \to 0
			\end{align}
			Therefore, $(x_n , y_n) \to (x , y)$ in $\mathbb{K}$. $(\cdot , \cdot) : X \times X \longrightarrow \mathbb{K}$ is continuous. 
		\end{proof}
	\end{proposition}

\newpage

\subsection{极化恒等式与平行四边形公式}
	本小节将说明赋范空间配备上\textbf{平行四边形公式}后即可成为内积空间, 并给出内积的定义式 -- \textbf{极化恒等式}. 首先来给出内积空间的\textbf{极化恒等式}. 
	
	\vspace{1em}
	
	\begin{proposition}\label{prop 3.2.2}
		\textbf{[极化恒等式]}. \\
		设$(X , (\cdot , \cdot))$ 为定义在数域$\mathbb{K}$ 上的内积空间, 则 
		\begin{enumerate}
			\item[(\rmnum{1})] If $\mathbb{K} = \R$, then 
			\[ (x , y) = \dfrac{1}{4} \Big( \Vert x + y \Vert^2 - \Vert x - y \Vert^2 \Big) , \,\, \forall x , y \in X \]
			
			\item[(\rmnum{2})] If $\mathbb{K} = \C$, then
			\[ (x , y) = \dfrac{1}{4} \Big( \Vert x + y \Vert^2 - \Vert x - y \Vert^2 \Big) + \dfrac{i}{4} \Big( \Vert x + iy \Vert^2 - \Vert x - iy \Vert^2 \Big) , \,\, \forall x , y \in X \]
		\end{enumerate}
		
		\vspace{6em}
		
		\begin{proof}
			当$\mathbb{K} = \C$ 时, 
			\begin{align}
				\frac{1}{4} \Big( \Vert x + y \Vert^2 - \Vert x - y \Vert^2 \Big) 
				&= Re(x , y) \\
				\frac{i}{4} \Big( \Vert x + iy \Vert^2 - \Vert x - iy \Vert^2 \Big) 
				&= i Re(x , iy) , \,\, \forall x , y \in X
			\end{align}
			而
			\[ Re(x , iy) = Re \Big( \bar{i} (x , y) \Big) = Im(x , y) , \,\, \forall x , y \in X \]
			Therefore, 
			\begin{align}
				&\frac{1}{4} \Big( \Vert x + y \Vert^2 - \Vert x - y \Vert^2 \Big) + \dfrac{i}{4} \Big( \Vert x + iy \Vert^2 - \Vert x - iy \Vert^2 \Big) \\
				&= Re(x , y) + i Im(x , y) \\
				&= (x , y) , \,\, \forall x , y \in X
			\end{align}
		\end{proof}
	\end{proposition}

	\vspace{4em}
	
	下面我们将说明, 对于$B^*$ 空间$(X , \Vert \cdot \Vert)$, 若其范数满足\textbf{平行四边形公式}, 则可引入内积$(\cdot , \cdot)$, $\st$
	\[ \sqrt{(x , x)} = \Vert x \Vert , \,\, \forall x \in X \]
	
	\newpage
	
	\begin{thm}\label{thm 3.2.1}
		\textbf{[范数诱导内积]}. \\
		设$(X , \Vert \cdot \Vert) \in B^*$, 若其范数满足平行四边形公式, 即
		\[ \Vert x + y \Vert^2 + \Vert x - y \Vert^2 = 2 \Big( \Vert x \Vert^2 + \Vert y \Vert^2 \Big) , \,\, \forall x , y \in X \]
		则可定义由此范数诱导的内积$(\cdot , \cdot)$, $\st \sqrt{(x , x)} = \Vert x \Vert , \,\, \forall x \in X$, 使之成为内积空间. 
		
		\vspace{6em}
		
		\begin{proof}
			内积按照\textbf{极化恒等式 (Prop \ref{prop 3.2.2})}定义. \\
			下面我们只证明$\mathbb{K} = \R$ 的情形, $\mathbb{K} = \C$ 可类似证明. 即定义
			\[ (x , y) \coloneqq \frac{1}{4} \Big( \Vert x + y \Vert^2 - \Vert x - y \Vert^2 \Big) , \forall x , y \in X \]
			依次验证\textbf{内积的三条公理 (Def \ref{def 3.1.1})}, \textbf{正定性}及\textbf{共轭对称性}显然成立. \\ 
			下面分两方面来证明\textbf{关于第一变元线性性}:(对加法$\&$ 数乘封闭) 
			
			\vspace{1em}
			
			\begin{itemize}
				\item \textbf{对加法封闭}:Since
				\begin{align}
					(x , z) + (y , z) 
					&= \frac{1}{4} \Big( \Vert x + z \Vert^2 - \Vert x - z \Vert^2 + \Vert y + z \Vert^2 - \Vert y - z \Vert^2 \Big) \\
					&= \frac{1}{8} \Big[ 2 \Big( \Vert x + z \Vert^2 + \Vert y + z \Vert^2 \Big) - 2 \Big( \Vert x - z \Vert^2 + \Vert y - z \Vert^2 \Big) \Big]
				\end{align}
				By \textbf{平行四边形公式}, 
				\begin{align}
					(x , z) + (y , z) 
					&= \frac{1}{8} \Big[ 2 \Big( \Vert x + z \Vert^2 + \Vert y + z \Vert^2 \Big) - 2 \Big( \Vert x - z \Vert^2 + \Vert y - z \Vert^2 \Big) \Big] \\
					&= \frac{1}{8} \Big( \Vert x + y + 2z \Vert^2 + \Vert x - y \Vert^2 - \Vert x + y - 2z \Vert^2 - \Vert x - y \Vert^2 \Big) \\ 
					&= \frac{1}{8} \Big( \Vert x + y + 2z \Vert^2 - \Vert x + y - 2z \Vert^2 \Big) \\
					&= \frac{1}{2} (x + y , 2z) 
					= 2 \left( \frac{x + y}{2} , z \right) , \,\, \forall x , y , z \in X
				\end{align}
				Let $y = 0$, we get
				\[ (x , z) = 2 \left( \frac{x}{2} , z \right) , \,\, \forall x , z \in X \]
				Thus replace $x$ by $x + y$, 
				\[ (x , z) + (y , z) = 2 \left( \frac{x + y}{2} , z \right) = (x + y , z) , \,\, \forall x , y , z \in X \]
				i.e. 
				\[ (x , z) + (y , z) = (x + y , z) , \,\, \forall x , y , z \in X \]
				
				\newpage
				
				\item \textbf{对数乘封闭}:Fix $\forall x , y \in X$. Let
				\begin{align}
					f : \R &\longrightarrow \R \\
					t &\longmapsto f(t) = (tx , y)
				\end{align}
				只需证:$f(t)$ 为线性函数. \\
				By the previous result, 
				\[ f(t_1 + t_2) = f(t_1) + f(t_2) , \,\, \forall t_1 , t_2 \in \R \]
				Since $\Vert \cdot \Vert$ is continuous (\textbf{范数的连续性 (Def \ref{def 2.1.1})}), then $f$ is continuous. \\
				Thus $f$ 为线性函数, i.e.
				\[ f(t) = tf(1) , \,\, \forall t \in \R \]
				Therefore, 
				\[ (tx , y) = t(x , y) , \,\, \forall t \in \R \]
			\end{itemize}
			
			综上, 
			\[ (\alpha x + \beta y , z) = \alpha (x , z) + \beta (y , z) , \,\, \forall x , y , z \in X , \,\, \forall \alpha , \beta \in \R \]
		\end{proof}
	\end{thm}

	\vspace{6em}
	
	\begin{example}\label{ex 3.2.1}
		\textbf{[$l^p$ 及$L^p$ 内积空间]}. 
		\begin{center}
			\textbf{对于$l^p$ 空间 (Ex \ref{ex 1.3.1})及$L^p$ 空间, 平行四边形公式成立的充要条件均为$p = 2$}.
		\end{center}
		
		\vspace{2em}
		
		\begin{proof}
			充分性可按定义验证. 对于必要性, 只需给出$l^p$ 空间特例即可 ($l^p$ 空间可视作$L^p$ 空间的子空间(阶梯函数)). \\
			Let $x = (1 , 0 , 0 , \cdots) , y = (0 , 1 , 0 , \cdots) \in l^p$, then $\Vert x \Vert = \Vert y \Vert = 1$, 
			\[ \Vert x + y \Vert = \left( 1^p + 1^p \right)^{\tfrac{1}{p}} = 2^{\tfrac{1}{p}} , \,\, \Vert x - y \Vert = \left( 1^p + \left| -1 \right|^p \right)^{\tfrac{1}{p}} = 2^{\tfrac{1}{p}} \]
			Thus $\Vert x + y \Vert^2 + \Vert x - y \Vert^2 = 2 \Big( \Vert x \Vert^2 + \Vert y \Vert^2 \Big) \,\, \Rightarrow \,\, 2 \cdot 2^{\tfrac{2}{p}} = 2 \cdot 2 \,\, \Rightarrow \,\, p = 2$. 
		\end{proof}
	\end{example}

\newpage

\section{正交分解}
\subsection{正交补}
	这一节开始我们来探索内积空间的几何性质. 首先给出在高等代数中接触过的\textbf{正交}及\textbf{正交补}的概念. 
	
	\vspace{1em}
	
	\begin{defn}\label{def 3.3.1}
		对于内积空间$(X , (\cdot , \cdot))$ 中的两个元素$x , y \in X$, 如果
		\[ (x , y) = 0 \]
		则称$x$ 与$y$ \underline{\textcolor{blue}{\textbf{正交}}}. 设$M \subset X$ 为非空子集, 如果$\forall y \in M$, $(x , y) = 0$, 则称$x$ 与$M$ \underline{\textcolor{blue}{\textbf{正交}}}. \\
		同时, 对于内积空间$(X , (\cdot , \cdot))$ 的子集$M \subset X$, 定义其\underline{\textcolor{blue}{\textbf{正交补$M^{\perp}$}}} 如下:
		\[ M^{\perp} \coloneqq \left\{ x \in X \mid \forall y \in M , \,\, (x , y) = 0 \right\} \]
		即为与$M$ 中所有成员均正交的元素所构成的集合. 
		
		\vspace{2em}
		
		\begin{rmk}
			\begin{itemize}
				\item $\forall M \subset X$, $0 \in M^{\perp} \neq \varnothing$. 
				
				\vspace{1em}
				
				\item 正交补继承了补运算的\textbf{反包含性}, 即$\forall M_1 \subset M_2$, 有$M_{2}^{\perp} \subset M_{1}^{\perp}$. (根据定义容易验证) 
				
				\vspace{1em}
				
				\item $\forall M \subset X , \,\, M^{\perp} \cap M \subset \{ 0 \}$, 即$M$ 与$M^{\perp}$ 交集为0 或空集. 
				
				\vspace{1em}
				
				\item $M^{\perp}$ 为$X$ 的\textbf{闭线性子空间}, 即对\textbf{加法}、\textbf{数乘}和\textbf{极限}封闭. (并不依赖于$M$ 的线性性)
				
				\vspace{3em}
				
				\begin{proof}
					根据\textbf{内积关于第一变元的线性性(Def \ref{def 3.1.1})}, $\forall x , y \in M^{\perp}$, 
					\[ (\alpha x + \beta y , z) = \alpha (x , z) + \beta (y , z) = 0 , \,\, \forall z \in M , \,\, \forall \alpha , \beta \in \mathbb{K} \]
					于是$\alpha x + \beta y \in M^{\perp} , \,\, \forall \alpha , \beta \in \mathbb{K}$, 即为线性子空间. \\
					$\forall \{ x_n \}_{n = 1}^{\infty} \subset M^{\perp}$ with $x_n \overset{\Vert \cdot \Vert}{\to} x_0 \in X$. 根据\textbf{内积的连续性 (Prop \ref{prop 3.2.1})}, 
					\[ (x_0 , y) 
					= \left( \lim_{n \to \infty} x_n , y \right) 
					= \lim_{n \to \infty} \left( x_n , y \right) 
					= 0 , \,\, \forall y \in M \]
					Thus $x_0 \in M^{\perp}$. Therefore, $M^{\perp}$ 对极限运算封闭, 即为闭线性子空间. 
				\end{proof}
				
				\newpage
				
				\item 根据$M^{\perp}$ 为$X$ 的闭线性子空间, 不难得到将$M$ 扩张为线性空间后$M^{\perp}$ 不变, 即
				\[ M^{\perp} = span(M)^{\perp} = \overline{span(M)}^{\perp} , \,\, \forall M \subset X \]
				
				\vspace{3em}
				
				\begin{proof}
					分别证明两个等式:\\
					$M^{\perp} = span(M)^{\perp}$:Trivial. \\
					$span(M)^{\perp} = \overline{span(M)}^{\perp}$:By \textbf{反包含性}, $\overline{span(M)}^{\perp} \subset span(M)^{\perp}$. \\
					Fix $x \in span(M)^{\perp}$. $\forall y \in \overline{span(M)}$. 
					\begin{enumerate}
						\item[(a)] If $y \in span(M)$, then since $x \in span(M)^{\perp}$, then $(x , y) = 0$. 
						
						\item[(b)] If $y \in \overline{span(M)} \, \backslash \, span(M)$, then $\exists \{ y_n \}_{n = 1}^{\infty} \subset span(M)$, $\st y_n \overset{\Vert \cdot \Vert}{\to} y$. \\
						By \textbf{内积的连续性 (Prop \ref{prop 3.2.1})}, 
						\[ (x , y) 
						= \left( x , \lim_{n \to \infty} y_n \right) 
						= \lim_{n \to \infty} (x , y_n) 
						= 0 \]
					\end{enumerate}
					Therefore, 
					\[ (x , y) = 0 , \,\, \forall y \in \overline{span(M)}^{\perp} \]
					$span(M)^{\perp} \subset \overline{span(M)}^{\perp} \,\, \Rightarrow \,\, span(M)^{\perp} = \overline{span(M)}^{\perp}$.
				\end{proof}
				
				\vspace{8em}
				
				\item $\forall M \subset X$, 其正交补空间$M^{\perp}$满足勾股定理, 即
				\[ \left\Vert \sum_{i = 1}^{n} x_i \right\Vert^2 = \sum_{i = 1}^n \Vert x_i \Vert^2 , \,\, \forall x_i \in M^{\perp} , \,\, \forall i = 1 \sim n \]
				
				\vspace{3em}
				
				\begin{proof}
					只需对$n = 2$ 情形证明, 即$\forall x , y \in M^{\perp}$, WTS:$\Vert x + y \Vert^2 = \Vert x \Vert^2 + \Vert y \Vert^2$. \\
					Since $x , y \in M^{\perp}$, $(x , y) = 0$, then
					\begin{align}
						\Vert x + y \Vert^2 
						= (x + y , x + y) 
						&= (x , x) + 2Re(x , y) + (y , y) \\
						&= (x , x) + (y , y) \\
						&= \Vert x \Vert^2 + \Vert y \Vert^2 , \,\, \forall x , y \in M^{\perp}
					\end{align}
					再由归纳法容易证明原命题. 
				\end{proof}
			\end{itemize}
		\end{rmk}
	\end{defn}

\newpage

\subsection{内积空间的严格凸性}
	回顾\textbf{严格凸 (Def \ref{def 2.3.1})} 的定义, 下面我们说明内积空间均为严格凸的$B^*$ 空间. 
	
	\vspace{1em}
	
	\begin{proposition}\label{prop 3.3.1}
		\textbf{[内积空间的严格凸性]}. 
		\begin{center}
			\textbf{内积空间$(X , (\cdot , \cdot))$ 必严格凸}.
		\end{center}
		
		\vspace{4em}
		
		\begin{proof}
			$\forall x , y \in X$ with $\Vert x \Vert = \Vert y \Vert = 1$, $x \neq y$. Fix $x$ and $y$. \\
			Let
			\begin{align}
				f : [0 , 1] &\longrightarrow \R_{\geq 0} \\
				\lambda &\longmapsto f(\lambda) = \Big\Vert (1 - \lambda) x + \lambda y \Big\Vert
			\end{align}
			By \textbf{范数的连续性 (Def \ref{def 2.1.1})}, $f \in C[0 , 1]$. Since 平行四边形公式 can be represented as
			\begin{align}
				\Vert x + y \Vert^2 + \Vert x - y \Vert^2 
				&= 2 \Big( \Vert x \Vert^2 + \Vert y \Vert^2 \Big) \\
				\Leftrightarrow \,\, 
				\left\Vert \frac{x + y}{2} \right\Vert^2 + \left\Vert \frac{x - y}{2} \right\Vert^2 
				&= \frac{1}{2} \Big( \left\Vert x \right\Vert^2 + \left\Vert y \right\Vert^2 \Big)
			\end{align}
			Take 
			\[ x \Rightarrow (1 - \lambda)x + \lambda y , \,\, y \Rightarrow (1 - \mu)x + \mu y , \,\, \forall 0 \leq \lambda < \mu \leq 1  \]
			Then by \textbf{平行四边形公式}, since $\Big\Vert (\mu - \lambda)(x - y) \Big\Vert > 0$, 
			\begin{align}
				\left\Vert \left( 1 - \frac{\lambda + \mu}{2} \right) x + \frac{\lambda + \mu}{2} y  \right\Vert^2 
				&= \frac{1}{2} \Big( \Big\Vert (1 - \lambda)x + \lambda y \Big\Vert^2 + \Big\Vert (1 - \mu)x + \mu y \Big\Vert^2 \Big) - \Big\Vert (\mu - \lambda)(x - y) \Big\Vert^2 \\
				&< \frac{1}{2} \Big( \Big\Vert (1 - \lambda)x + \lambda y \Big\Vert^2 + \Big\Vert (1 - \mu)x + \mu y \Big\Vert^2 \Big)
			\end{align}
			i.e. 
			\[ f\left( \frac{\lambda + \mu}{2} \right) < \frac{1}{2} \Big( f\left( \lambda \right) + f(\mu) \Big) , \,\, \forall 0 \leq \lambda < \mu \leq 1 \]
			Since $f \in C[0 , 1]$ continuous, then $f$ is \textbf{strictly convex (严格凸)}. \\
			With $f(0) = f(1) = 1$, we can easily proof that $X$ is also \textbf{strictly convex}. 
			\begin{center}
				(对$\forall \lambda_0 \in (0 , 1)$, 可用二分法得到一系列区间端点$\lambda_n \to \lambda_0$, 从而得到严格不等式$f(\lambda_0) < 1$)
			\end{center}
		\end{proof}
	\end{proposition}

\newpage

\subsection{Hilbert空间的闭凸子集 (最佳逼近问题)}
	回顾我们曾在介绍赋范空间严格凸性时, 给出过\textbf{对于严格凸赋范空间, 其空间上一点到有限维子空间的最佳逼近问题解的存在唯一性 (Prop \ref{prop 2.3.2})}. 而对于\textbf{Hilbert空间}, 我们同样有类似的定理, 并且可以用更一般的\textbf{“闭凸子集”}代替\textbf{“有限维子空间”}的条件. 
	
	\vspace{1em}
	
	首先给出\textbf{Hilbert空间}的定义. 
	
	\vspace{1em}
	
	\begin{defn}\label{def 3.3.2}
		完备的内积空间$(X , (\cdot , \cdot))$ 称为\underline{\textcolor{blue}{\textbf{Hilbert空间}}}, 记$X \in \mathcal{H}$. 
		
		\begin{rmk}
			此处的“完备”指的是由内积$(\cdot , \cdot)$ 所诱导的范数$\Vert \cdot \Vert$ 再诱导的度量$\rho$ 是完备的. 
		\end{rmk}
	\end{defn}
	
	\vspace{4em}
	
	下面给出Hilbert空间上最佳逼近问题解的存在唯一性定理. 
	
	\vspace{1em}
	
	\begin{thm}\label{thm 3.3.1}
		\textbf{[Hilbert空间最佳逼近问题解的存在唯一性]}. \\
		设$(X , (\cdot , \cdot)) \in \mathcal{H}$, $M \subset X$ 为闭凸子集, then for $\forall x \in X$, $\exists$ 唯一的$x_0 \in M$, $\st$
		\[ \Vert x - x_0 \Vert = dist(x , M) \]
		
		\vspace{2em}
		
		\begin{rmk}
			\textbf{Hilbert空间的完备性}及\textbf{闭凸子集}保证了解的\textbf{存在性}, 而\textbf{内积空间的严格凸性 (Prop \ref{prop 3.3.1})}则自动保证了解的\textbf{唯一性}.
		\end{rmk}
		
		\vspace{6em}
		
		\begin{proof}
			\begin{itemize}
				\item \textbf{存在性}:Since $dist(x , M) = \underset{y \in M}{\inf} \Vert x - y \Vert$, then for $\forall \epsilon = \dfrac{1}{n}$, $\exists x_n \in M$, $\st$
				\[ dist(x , M) \leq \Vert x - x_n \Vert < dist(x , M) + \frac{1}{n} , \,\, \forall n \in \N \]
				i.e. 
				\[ \Vert x - x_n \Vert \to dist(x , M) \,\, \text{as} \,\, n \to \infty \]
				下证$\{ x_n \}_{n = 1}^{\infty} \subset M$ 为Cauchy sequence:\\
				By \textbf{平行四边形公式}, 
				\begin{align}
					\Vert x_n - x_m \Vert^2 
					&= \Big\Vert \left( x_n - x \right) + \left( x - x_m \right) \Big\Vert^2 \\
					&= 2 \Big( \Vert x_n - x \Vert^2 + \Vert x - x_m \Vert^2 \Big) - \Big\Vert 2x - \left( x_m + x_n \right) \Big\Vert^2 \\
					&= 2 \Big( \Vert x_n - x \Vert^2 + \Vert x - x_m \Vert^2 \Big) - 4 \Big\Vert x - \frac{x_m + x_n}{2} \Big\Vert^2 
				\end{align}
				Since $M \subset X$ 为闭凸子集, $x_m , x_n \in X$, then $\dfrac{x_m + x_n}{2} \in M$, thus $\Big\Vert x - \dfrac{x_m + x_n}{2} \Big\Vert \geq dist(x , M)$, 
				\begin{align}
					\Vert x_n - x_m \Vert^2 
					&= 2 \Big( \Vert x_n - x \Vert^2 + \Vert x - x_m \Vert^2 \Big) - 4 \Big\Vert x - \frac{x_m + x_n}{2} \Big\Vert^2 \\
					&\leq 2 \Big( \Vert x_n - x \Vert^2 + \Vert x - x_m \Vert^2 \Big) - 4 dist(x , M)^2
				\end{align}
				Since $dist(x , M) \leq \Vert x - x_n \Vert < dist(x , M) + \dfrac{1}{n}$, then for $\forall \epsilon > 0$, $\exists N_\epsilon \in \N$, $\st$
				\[ \Vert x_n - x_m \Vert^2 \leq 4\epsilon^2 , \,\, \forall n , m \geq N_\epsilon \]
				Thus $\{ x_n \}_{n = 1}^{\infty} \subset M$ is a Cauchy sequence. \\
				Since $X \in \mathcal{H}$ is complete, then $\exists x_0 \in X$, $\st$
				\[ x_n \overset{\Vert \cdot \Vert}{\to} x_0 \]
				Since $M \subset X$ is closed, $\{ x_n \}_{n = 1}^{\infty} \subset M$, then $x_n \to x_0 \in M$. By \textbf{范数的连续性 (Def \ref{def 2.1.1})}, 
				\[ \Vert x - x_0 \Vert 
				= \Big\Vert x - \lim_{n \to \infty} x_n \Big\Vert 
				= \lim_{n \to \infty} \Vert x - x_n \Vert 
				= dist(x , M) \]
				
				\vspace{8em}
				
				\item \textbf{唯一性}:同\textbf{Prop \ref{prop 2.3.2} 唯一性证明}. 即利用\textbf{内积空间的严格凸性 (Prop \ref{prop 3.3.1})}.
			\end{itemize}
		\end{proof}
	\end{thm}

\newpage

\subsection{投影定理 (正交分解)}
	下面我们给出\textbf{Hilbert空间上的投影定理 (正交分解)}, 它事实上给出了Hilbert空间\textbf{直和分解}的理论依据. 
	
	\vspace{1em}
	
	\begin{thm}\label{thm 3.3.2}
		\textbf{[Hilbert空间上的投影定理]}. \\
		设$(X , (\cdot , \cdot)) \in \mathcal{H}$, $M \subset X$ 为闭(线性)子空间, then for $\forall x \in X$, $\exists$ 唯一的$x_0 \in M$, $x_1 \in M^{\perp}$, $\st$
		\[ x = x_0 + x_1 \]
		
		\vspace{2em}
		
		\begin{rmk}
			事实上该投影定理给出Hilbert空间的直和分解, 即$\forall$ 闭子空间$M \subset X$, 
			\[ X = M \oplus M^{\perp} \]
			更一般地, 对于$\forall$ 子集$M \subset X$, 根据$M^{\perp} = span(M)^{\perp} = \overline{span(M)}^{\perp}$ \textbf{(Def \ref{def 3.3.1})}, 
			\[ X = \overline{span(M)} \oplus M^{\perp} \]
		\end{rmk}
	
		\vspace{4em}
		
		\begin{proof}
			\begin{itemize}
				\item \textbf{存在性}:Since $M \subset X$ 为闭子空间, 故为线性空间, 同时为闭凸子集, then by \textbf{Thm \ref{thm 3.3.1}}, \\
				$\exists x_0 \in M$, $\st$
				\[ \Vert x - x_0 \vert = dist(x , M) \]
				下面证明$x - x_0 \in M^{\perp}$:$\forall y \in M$, 
				\[ \Vert x - x_0 - y \Vert^2 = \Vert x - x_0 \Vert^2 + \Vert y \Vert^2 - 2 Re(x - x_0 , y) \]
				Since $M$ is linear, then $x_0 + y \in M$, 
				\[ \Vert x - x_0 - y \Vert^2 = \Big\Vert x - (x_0 + y) \Big\Vert^2 \geq dist(x , M)^2 = \Vert x - x_0 \Vert^2 \]
				i.e.
				\[ \Vert y \Vert^2 \geq 2 Re(x - x_0 , y) , \,\, \forall y \in M \]
				Fix $y \in M$. Since $M$ is linear, then $ty \in M , \,\, \forall t \in \R$, replace $y$ as $ty$, 
				\[ t^2 \Vert y \Vert^2 \geq 2t \, Re(x - x_0 , y) , \,\, \forall t \in \R \]
				i.e.
				\[ \Vert y \Vert^2 \cdot t^2 - 2 \, Re(x - x_0 , y) \cdot t \geq 0 , \,\, \forall t \in \R \]
				Calculate 上述二次函数判别式, we have
				\[ \Delta = 4 Re(x - x_0 , y)^2 \leq 0 \,\, \Rightarrow \,\, Re(x - x_0 , y) = 0 \]
				Similarly, replace $y$ as $ity \in M , \,\, \forall t \in \R$, 
				\begin{align}
					t^2 \Vert y \Vert^2 
					\geq 2t \, Re(x - x_0 , iy) 
					&= 2t \, Re(\overline{i} (x - x_0 , y)) \\
					&= 2t \, Re \left( \frac{(x - x_0 , y)}{i} \right) \\
					&= 2t \, Im(x - x_0 , y)
				\end{align}
				Calculate 判别式, we can get 
				\[ Im(x - x_0 , y) = 0 \]
				Therefore, 
				\[ (x - x_0 , y) = Re(x - x_0 , y) + i \, Im(x - x_0 , y) = 0 , \,\, \forall y \in M \]
				i.e. 
				\[ x - x_0 \in M^{\perp} \]
				Let $x_1 = x - x_0 \in M^{\perp}$, thus $x = x_0 + x_1$ for some $x_0 \in M , x_1 \in M^{\perp}$. 
				
				\vspace{10em}
				
				\item \textbf{唯一性}:Suppose $x = x_0 + x_1 = y_0 + y_1$, where $x_0 , y_0 \in M , \,\, x_1 , y_1 \in M^{\perp}$. Then
				\[ x_0 - y_0 = y_1 - x_1 \]
				Since both $M$ and $M^{\perp}$ are linear, then $x_0 - y_0 \in M$ and $y_1 - x_1 \in M^{\perp}$. Thus
				\[ x_0 - y_0 = y_1 - x_1 \in M \cap M^{\perp} \]
				Since $M \cap M^{\perp} \subset \{ 0 \}$ \textbf{(Def \ref{def 3.3.1})}, then 
				\[ x_0 = y_0 , \,\, x_1 = y_1 \]
			\end{itemize}
		\end{proof}
	\end{thm}

\newpage

\subsection{正交补的性质}
	这一小节我们来补充\textbf{Hilbert空间中正交补的性质}. 
	
	\vspace{1em}
	
	\begin{proposition}\label{prop 3.3.2}
		\textbf{[Hilbert空间中正交补的性质]}. \\
		设$(X , (\cdot , \cdot)) \in \mathcal{H}$, $M \subset X$ 为闭子空间, 则
		
		\vspace{1em}
		
		\begin{enumerate}
			\item[(\rmnum{1})] $X = M \oplus M^{\perp}$. 更一般地, 对于$\forall$ 子集$M \subset X$, 
			\[ X = \overline{span(M)}^{\perp} \oplus M^{\perp} \]
			
			\item[(\rmnum{2})] 若$M \subsetneqq X$ 为真闭子空间, 则$M^{\perp} \neq \{ 0 \}$. 
			
			\item[(\rmnum{3})] $M = \left( M^{\perp} \right)^{\perp}$ 
			
			\item[(\rmnum{4})] 设$M \subset X$ 为子空间, 且$M^{\perp} = \{ 0 \}$, 则$\overline{M} = X$. 
		\end{enumerate}
		
		\vspace{6em}
		
		\begin{proof}
			\begin{enumerate}
				\item[(\rmnum{3})] \textbf{Claim}:$\forall$ 闭子空间$M \subset X$, $M \subset \left( M^{\perp} \right)^{\perp}$. 
				\begin{itemize}
					\item $\forall x \in M$. Fix $x$. $\forall y \in M^{\perp}$, then $(y , z) = 0 , \,\, \forall z \in M$. Thus
					\[ (x , y) = 0 , \,\, \forall y \in M^{\perp} \]
					Then $x \in \left( M^{\perp} \right)^{\perp} \,\, \Rightarrow \,\, M \subset \left( M^{\perp} \right)^{\perp}$. 
				\end{itemize}
				By \textbf{Thm \ref{thm 3.3.2}} (即本命题(\rmnum{1})), 
				\[ X = M \oplus M^{\perp} \subset \left( M^{\perp} \right)^{\perp} \oplus M^{\perp} \]
				Thus $X = M \oplus M^{\perp} = \left( M^{\perp} \right)^{\perp} \oplus M^{\perp}$. Therefore, $M = \left( M^{\perp} \right)^{\perp}$. 
				
				\vspace{5em}
				
				\item[(\rmnum{4})] Since $M \subset \overline{M}$, then 
				\[ X = M \oplus M^{\perp} \subset \overline{M} \oplus M^{\perp} \]
				Thus $X = \overline{M} \oplus M^{\perp}$. Since $M^{\perp} = \{ 0 \}$, then $X = \overline{M}$.
			\end{enumerate}
		\end{proof}
	\end{proposition}
	
















	%  ############################
	\ifx\allfiles\undefined
\end{document}
\fi