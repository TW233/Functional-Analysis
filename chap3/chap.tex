\ifx\allfiles\undefined
\documentclass[12pt, a4paper,oneside, UTF8]{ctexbook}
\usepackage[dvipsnames]{xcolor}
\usepackage{mathtools}   % 数学公式
\usepackage{amsthm}    % 定理环境
\usepackage{amssymb}   % 更多公式符号
\usepackage{graphicx}  % 插图
%\usepackage{mathrsfs}  % 数学字体
%\usepackage{newtxtext,newtxmath}
%\usepackage{arev}
\usepackage{kmath,kerkis}
\usepackage{newtxtext}
\usepackage{bbm}
\usepackage{enumitem}  % 列表
\usepackage{geometry}  % 页面调整
%\usepackage{unicode-math}
\usepackage[colorlinks,linkcolor=black]{hyperref}

\usepackage{wrapfig}


\usepackage{ulem}	   % 用于更多的下划线格式,
					   % \uline{}下划线,\uuline{}双下划线,\uwave{}下划波浪线,\sout{}中间删除线,\xout{}斜删除线
					   % \dashuline{}下划虚线,\dotuline{}文字底部加点


\graphicspath{ {flg/},{../flg/}, {config/}, {../config/} }  % 配置图形文件检索目录
\linespread{1.5} % 行高

% 页码设置
\geometry{top=25.4mm,bottom=25.4mm,left=20mm,right=20mm,headheight=2.17cm,headsep=4mm,footskip=12mm}

% 设置列表环境的上下间距
\setenumerate[1]{itemsep=5pt,partopsep=0pt,parsep=\parskip,topsep=5pt}
\setitemize[1]{itemsep=5pt,partopsep=0pt,parsep=\parskip,topsep=5pt}
\setdescription{itemsep=5pt,partopsep=0pt,parsep=\parskip,topsep=5pt}

% 定理环境
% ########## 定理环境 start ####################################
\theoremstyle{definition}
\newtheorem{defn}{\indent 定义}[section]

\newtheorem{lemma}{\indent 引理}[section]    % 引理 定理 推论 准则 共用一个编号计数
\newtheorem{thm}[lemma]{\indent 定理}
\newtheorem{corollary}[lemma]{\indent 推论}
\newtheorem{criterion}[lemma]{\indent 准则}

\newtheorem{proposition}{\indent 命题}[section]
\newtheorem{example}{\indent \color{SeaGreen}{例}}[section] % 绿色文字的 例 ,不需要就去除\color{SeaGreen}{}
\newtheorem*{rmk}{\indent \color{red}{注}}

% 两种方式定义中文的 证明 和 解 的环境:
% 缺点:\qedhere 命令将会失效【技术有限,暂时无法解决】
\renewenvironment{proof}{\par\textbf{证明.}\;}{\qed\par}
\newenvironment{solution}{\par{\textbf{解.}}\;}{\qed\par}

% 缺点:\bf 是过时命令,可以用 textb f等替代,但编译会有关于字体的警告,不过不影响使用【技术有限,暂时无法解决】
%\renewcommand{\proofname}{\indent\bf 证明}
%\newenvironment{solution}{\begin{proof}[\indent\bf 解]}{\end{proof}}
% ######### 定理环境 end  #####################################

% ↓↓↓↓↓↓↓↓↓↓↓↓↓↓↓↓↓ 以下是自定义的命令  ↓↓↓↓↓↓↓↓↓↓↓↓↓↓↓↓

% 用于调整表格的高度  使用 \hline\xrowht{25pt}
\newcommand{\xrowht}[2][0]{\addstackgap[.5\dimexpr#2\relax]{\vphantom{#1}}}

% 表格环境内长内容换行
\newcommand{\tabincell}[2]{\begin{tabular}{@{}#1@{}}#2\end{tabular}}

% 使用\linespread{1.5} 之后 cases 环境的行高也会改变,重新定义一个 ca 环境可以自动控制 cases 环境行高
\newenvironment{ca}[1][1]{\linespread{#1} \selectfont \begin{cases}}{\end{cases}}
% 和上面一样
\newenvironment{vx}[1][1]{\linespread{#1} \selectfont \begin{vmatrix}}{\end{vmatrix}}

\def\d{\textup{d}} % 直立体 d 用于微分符号 dx
\def\R{\mathbb{R}} % 实数域
\def\N{\mathbb{N}} % 自然数域
\def\C{\mathbb{C}} % 复数域
\def\Z{\mathbb{Z}} % 整数环
\def\Q{\mathbb{Q}} % 有理数域
\newcommand{\bs}[1]{\boldsymbol{#1}}    % 加粗,常用于向量
\newcommand{\ora}[1]{\overrightarrow{#1}} % 向量

% 数学 平行 符号
\newcommand{\pll}{\kern 0.56em/\kern -0.8em /\kern 0.56em}

% 用于空行\myspace{1} 表示空一行 填 2 表示空两行  
\newcommand{\myspace}[1]{\par\vspace{#1\baselineskip}}

%s.t. 用\st就能打出s.t.
\DeclareMathOperator{\st}{s.t.}

%罗马数字 \rmnum{}是小写罗马数字, \Rmnum{}是大写罗马数字
\makeatletter
\newcommand{\rmnum}[1]{\romannumeral #1}
\newcommand{\Rmnum}[1]{\expandafter@slowromancap\romannumeral #1@}
\makeatother
\begin{document}
	% \title{{\Huge{\textbf{$Functional \,\, Analysis$}}}\footnote{参考书籍:\\
			\hspace*{4em} \textbf{《Linear and Nonlinear Functional Analysis with Applications》 -- Philippe G. Ciarlet} \\
			\hspace*{4em} \textbf{《Real  Analysis -- Modern Techniques and Their Applications》 -- Gerald  B.  Folland} \\
			\hspace*{4em} \textbf{《Functional Analysis -- Introduction to Further Topics in Analysis》 -- Elias M. Stein} \\
			\hspace*{4em} \textbf{《泛函分析讲义》 -- 张恭庆、林源渠} 
			}}
\author{$-TW-$}
\date{\today}
\maketitle                   % 在单独的标题页上生成一个标题

\thispagestyle{empty}        % 前言页面不使用页码
\begin{center}
	\Huge\textbf{序}
\end{center}


\vspace*{3em}
\begin{center}
	\large{\textbf{天道几何,万品流形先自守;}}\\
	
	\large{\textbf{变分无限,孤心测度有同伦。}}
\end{center}

\vspace*{3em}
\begin{flushright}
	\begin{tabular}{c}
		\today \\ \small{\textbf{长夜伴浪破晓梦,梦晓破浪伴夜长}}
	\end{tabular}
\end{flushright}


\newpage                      % 新的一页
\pagestyle{plain}             % 设置页眉和页脚的排版方式(plain:页眉是空的,页脚只包含一个居中的页码)
\setcounter{page}{1}          % 重新定义页码从第一页开始
\pagenumbering{Roman}         % 使用大写的罗马数字作为页码
\tableofcontents              % 生成目录

\newpage                      % 以下是正文
\pagestyle{plain}
\setcounter{page}{1}          % 使用阿拉伯数字作为页码
\pagenumbering{arabic}
\setcounter{chapter}{0}    % 设置 -1 可作为第零章绪论从第零章开始 
	\else
	\fi
	%  ############################ 正文部分
\chapter{内积空间}

\section{内积空间}
	在先前的学习中, 我们已经得到了\textbf{度量空间}、\textbf{赋范空间}的概念, 它们的性质都很好, 但相比于我们所熟知的\textbf{欧氏空间$\R^n$}, 尤其是在\textbf{几何性质}上还存在着较大的差距. 这一章我们将给出\textbf{内积空间}的概念和性质, 并说明, 其与\textbf{赋范空间}之间只相差了一条\textbf{平行四边形公式}. 
	
	\vspace{1em}
	
	\begin{defn}\label{def 3.1.1}
		设$X$ 为定义在域$\mathbb{F}$ 上的线性空间, 如果映射$(\cdot , \cdot)$
		\[ (\cdot , \cdot) : X \times X \longrightarrow \mathbb{K} \]
		满足如下三条性质:
		\begin{enumerate}
			\item \textbf{[正定性]}. 
			\[ (x , x) \geq 0 , \,\, \forall x \in X \hspace*{4em} \Big( (x , x) = 0 \,\, \Leftrightarrow \,\, x = 0 \Big) \]
			
			\item \textbf{[共轭对称性]}. 
			\[ (x , y) = \overline{(y , x)} , \,\, \forall x , y \in X \]
			
			\item \textbf{[关于第一变元线性性]}. 
			\[ (\alpha x_1 + \beta x_2 , y) = \alpha (x_1 , y) + \beta (x_2 , y) , \,\, \forall x_1 , x_2 , y \in X , \,\, \forall \alpha , \beta \in \mathbb{F} \]
		\end{enumerate}
		则称$(\cdot , \cdot)$ 为$X$ 上的\underline{\textcolor{blue}{\textbf{内积}}}, $(X , (\cdot , \cdot))$ 称为\underline{\textcolor{blue}{\textbf{内积空间}}}. 
		
		\vspace{4em}
		
		\begin{rmk}
			\begin{itemize}
				\item $(x , 0) = (0 , x) = 0 , \,\, \forall x \in X$
				
				\vspace{3em}
				
				\item 内积关于\textbf{第二变元}具有\textbf{共轭线性性}:
				\[ (x , \alpha y_1 + \beta y_2) = \overline{\alpha} (x , y_1) + \overline{\beta} (x , y_2) , \,\, \forall x , y_1 , y_2 \in X , \,\, \forall \alpha , \beta \in \mathbb{F} \]
				
				\newpage
				
				\begin{proof}
					根据\textbf{共轭对称性}及\textbf{第一变元线性性}, $\forall x , y_1 , y_2 \in X , \,\, \forall \alpha , \beta \in \mathbb{F}$, 
					\[
						(x , \alpha y_1 + \beta y_2) 
						= \overline{(\alpha y_1 + \beta y_2 , x)} 
						= \overline{\alpha} \overline{(y_1  ,x)} + \overline{\beta} \overline{(y_2 , x)} 
						= \overline{\alpha} (x , y_1) + \overline{\beta} (x , y_2)
					\]
				\end{proof}
				
				\vspace{4em}
				
				\item 事实上, 内积可\textbf{诱导范数}, 即\textbf{内积空间为一类特殊的$B^*$ 空间}. 即
				\begin{center}
					\textbf{内积空间} $\,\, \Rightarrow \,\,$ \textbf{赋范空间} \hspace*{1em} , \hspace*{1em} \textbf{赋范空间} $\,\, \not\Rightarrow \,\,$ \textbf{内积空间}
				\end{center}
				后续我们将说明, 赋范空间只需再满足一条平行四边形公式, 即可扩充为内积空间. \\
				对于一般的内积空间$(X , (\cdot , \cdot))$, 我们总是诱导范数$\Vert \cdot \Vert$ 定义如下:
				\[ \Vert x \Vert = \sqrt{(x , x)} , \,\, \forall x \in X \]
				下面证明该定义满足\textbf{范数的三条公理 (Def \ref{def 2.1.1})}:
				
				\vspace{6em}
				
				\begin{proof}
					正定性不难验证. 对于绝对齐性, 根据\textbf{第一变元线性性}及\textbf{第二变元共轭线性性}, 
					\[ \Vert k x \Vert 
					= \sqrt{(k x , kx)} 
					= \sqrt{k(x , kx)} 
					= \sqrt{k \overline{k} (x , x)} 
					= \left| k \right| \sqrt{(x , x)} 
					= \left| k \right| \cdot \Vert x \Vert , \,\, \forall x \in X , \,\, \forall k \in \mathbb{F} \]
					而对于三角不等式, 根据我们接下来马上介绍的\textbf{Cauchy-Schwarz's Inequality (Thm \ref{thm 3.1.1})}, 
					\[ Re (x , y) \leq \Big| (x , y) \Big| \leq \Vert x \Vert \cdot \Vert y \Vert , \,\, \forall x , y \in X \]
					Thus
					\begin{align}
					\Vert x + y \Vert^2 
					= (x + y , x + y) 
					&= (x , x) + 2 Re(x , y) + (y , y) \\
					&\leq (x , x) + 2 \Vert x \Vert \cdot \Vert y \Vert + (y , y) \\
					&= \left( \Vert x \Vert + \Vert y \Vert \right)^2 , \,\, \forall x , y \in X 
					\end{align}
					i.e. 
					\[ \Vert x + y \Vert \leq \Vert x \Vert + \Vert y \Vert , \,\, \forall x , y \in X \]
					Therefore, $\Vert \cdot \Vert$ is a norm defined on $X$.
				\end{proof}
			\end{itemize}
		\end{rmk}
	\end{defn}

\newpage

\subsection{Cauchy-Schwarz's Inequality}
	下面我们将介绍大名鼎鼎的\textbf{Cauchy-Schwarz's Inequality}, 其在一般的内积空间中均成立, 并为内积空间中\textbf{“角度”}这一几何概念提供了理论支撑. 
	
	\vspace{1em}
	
	\begin{thm}\label{thm 3.1.1}
		\textbf{[Cauchy-Schwarz's Inequality]}. \\
		Suppose $(X , (\cdot , \cdot))$ be an inner product space. Let $\Vert x \Vert = \sqrt{(x , x)} , \,\, \forall x \in X$. Then 
		\[ \Big| (x , y) \Big| \leq \Vert x \Vert \cdot \Vert y \Vert , \,\, \forall x , y \in X \]
		且等号“$=$” 成立$\,\, \Leftrightarrow \,\, x , y$ 线性相关. 
		
		\vspace{4em}
		
		\begin{proof}
			
		\end{proof}
	\end{thm}
















	%  ############################
	\ifx\allfiles\undefined
\end{document}
\fi